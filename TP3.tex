% Options for packages loaded elsewhere
\PassOptionsToPackage{unicode}{hyperref}
\PassOptionsToPackage{hyphens}{url}
%
\documentclass[
]{article}
\usepackage{lmodern}
\usepackage{amssymb,amsmath}
\usepackage{ifxetex,ifluatex}
\ifnum 0\ifxetex 1\fi\ifluatex 1\fi=0 % if pdftex
  \usepackage[T1]{fontenc}
  \usepackage[utf8]{inputenc}
  \usepackage{textcomp} % provide euro and other symbols
\else % if luatex or xetex
  \usepackage{unicode-math}
  \defaultfontfeatures{Scale=MatchLowercase}
  \defaultfontfeatures[\rmfamily]{Ligatures=TeX,Scale=1}
\fi
% Use upquote if available, for straight quotes in verbatim environments
\IfFileExists{upquote.sty}{\usepackage{upquote}}{}
\IfFileExists{microtype.sty}{% use microtype if available
  \usepackage[]{microtype}
  \UseMicrotypeSet[protrusion]{basicmath} % disable protrusion for tt fonts
}{}
\makeatletter
\@ifundefined{KOMAClassName}{% if non-KOMA class
  \IfFileExists{parskip.sty}{%
    \usepackage{parskip}
  }{% else
    \setlength{\parindent}{0pt}
    \setlength{\parskip}{6pt plus 2pt minus 1pt}}
}{% if KOMA class
  \KOMAoptions{parskip=half}}
\makeatother
\usepackage{xcolor}
\IfFileExists{xurl.sty}{\usepackage{xurl}}{} % add URL line breaks if available
\IfFileExists{bookmark.sty}{\usepackage{bookmark}}{\usepackage{hyperref}}
\hypersetup{
  pdftitle={TP3 Real Estate},
  pdfauthor={Kaabi Linda \& Vhiny-Guilley Mombo},
  hidelinks,
  pdfcreator={LaTeX via pandoc}}
\urlstyle{same} % disable monospaced font for URLs
\usepackage[margin=1in]{geometry}
\usepackage{color}
\usepackage{fancyvrb}
\newcommand{\VerbBar}{|}
\newcommand{\VERB}{\Verb[commandchars=\\\{\}]}
\DefineVerbatimEnvironment{Highlighting}{Verbatim}{commandchars=\\\{\}}
% Add ',fontsize=\small' for more characters per line
\usepackage{framed}
\definecolor{shadecolor}{RGB}{248,248,248}
\newenvironment{Shaded}{\begin{snugshade}}{\end{snugshade}}
\newcommand{\AlertTok}[1]{\textcolor[rgb]{0.94,0.16,0.16}{#1}}
\newcommand{\AnnotationTok}[1]{\textcolor[rgb]{0.56,0.35,0.01}{\textbf{\textit{#1}}}}
\newcommand{\AttributeTok}[1]{\textcolor[rgb]{0.77,0.63,0.00}{#1}}
\newcommand{\BaseNTok}[1]{\textcolor[rgb]{0.00,0.00,0.81}{#1}}
\newcommand{\BuiltInTok}[1]{#1}
\newcommand{\CharTok}[1]{\textcolor[rgb]{0.31,0.60,0.02}{#1}}
\newcommand{\CommentTok}[1]{\textcolor[rgb]{0.56,0.35,0.01}{\textit{#1}}}
\newcommand{\CommentVarTok}[1]{\textcolor[rgb]{0.56,0.35,0.01}{\textbf{\textit{#1}}}}
\newcommand{\ConstantTok}[1]{\textcolor[rgb]{0.00,0.00,0.00}{#1}}
\newcommand{\ControlFlowTok}[1]{\textcolor[rgb]{0.13,0.29,0.53}{\textbf{#1}}}
\newcommand{\DataTypeTok}[1]{\textcolor[rgb]{0.13,0.29,0.53}{#1}}
\newcommand{\DecValTok}[1]{\textcolor[rgb]{0.00,0.00,0.81}{#1}}
\newcommand{\DocumentationTok}[1]{\textcolor[rgb]{0.56,0.35,0.01}{\textbf{\textit{#1}}}}
\newcommand{\ErrorTok}[1]{\textcolor[rgb]{0.64,0.00,0.00}{\textbf{#1}}}
\newcommand{\ExtensionTok}[1]{#1}
\newcommand{\FloatTok}[1]{\textcolor[rgb]{0.00,0.00,0.81}{#1}}
\newcommand{\FunctionTok}[1]{\textcolor[rgb]{0.00,0.00,0.00}{#1}}
\newcommand{\ImportTok}[1]{#1}
\newcommand{\InformationTok}[1]{\textcolor[rgb]{0.56,0.35,0.01}{\textbf{\textit{#1}}}}
\newcommand{\KeywordTok}[1]{\textcolor[rgb]{0.13,0.29,0.53}{\textbf{#1}}}
\newcommand{\NormalTok}[1]{#1}
\newcommand{\OperatorTok}[1]{\textcolor[rgb]{0.81,0.36,0.00}{\textbf{#1}}}
\newcommand{\OtherTok}[1]{\textcolor[rgb]{0.56,0.35,0.01}{#1}}
\newcommand{\PreprocessorTok}[1]{\textcolor[rgb]{0.56,0.35,0.01}{\textit{#1}}}
\newcommand{\RegionMarkerTok}[1]{#1}
\newcommand{\SpecialCharTok}[1]{\textcolor[rgb]{0.00,0.00,0.00}{#1}}
\newcommand{\SpecialStringTok}[1]{\textcolor[rgb]{0.31,0.60,0.02}{#1}}
\newcommand{\StringTok}[1]{\textcolor[rgb]{0.31,0.60,0.02}{#1}}
\newcommand{\VariableTok}[1]{\textcolor[rgb]{0.00,0.00,0.00}{#1}}
\newcommand{\VerbatimStringTok}[1]{\textcolor[rgb]{0.31,0.60,0.02}{#1}}
\newcommand{\WarningTok}[1]{\textcolor[rgb]{0.56,0.35,0.01}{\textbf{\textit{#1}}}}
\usepackage{graphicx,grffile}
\makeatletter
\def\maxwidth{\ifdim\Gin@nat@width>\linewidth\linewidth\else\Gin@nat@width\fi}
\def\maxheight{\ifdim\Gin@nat@height>\textheight\textheight\else\Gin@nat@height\fi}
\makeatother
% Scale images if necessary, so that they will not overflow the page
% margins by default, and it is still possible to overwrite the defaults
% using explicit options in \includegraphics[width, height, ...]{}
\setkeys{Gin}{width=\maxwidth,height=\maxheight,keepaspectratio}
% Set default figure placement to htbp
\makeatletter
\def\fps@figure{htbp}
\makeatother
\setlength{\emergencystretch}{3em} % prevent overfull lines
\providecommand{\tightlist}{%
  \setlength{\itemsep}{0pt}\setlength{\parskip}{0pt}}
\setcounter{secnumdepth}{-\maxdimen} % remove section numbering

\title{TP3 Real Estate}
\author{Kaabi Linda \& Vhiny-Guilley Mombo}
\date{13/11/2020}

\begin{document}
\maketitle

Dans cette étude on utilise le dataset ``RealEstate.csv'' pour expliquer
les hauts et les bas prix en utilisants des regressions logistiques.
\#\# 1) Chargemment des données et differents packages

\hypertarget{loading-packages}{%
\subsection{loading packages}\label{loading-packages}}

\begin{Shaded}
\begin{Highlighting}[]
\KeywordTok{library}\NormalTok{(Metrics)}
\KeywordTok{library}\NormalTok{(ggplot2)}
\KeywordTok{library}\NormalTok{(corrplot)}
\end{Highlighting}
\end{Shaded}

\begin{verbatim}
## corrplot 0.84 loaded
\end{verbatim}

\begin{Shaded}
\begin{Highlighting}[]
\KeywordTok{library}\NormalTok{(bayestestR)}
\end{Highlighting}
\end{Shaded}

\begin{verbatim}
## 
## Attaching package: 'bayestestR'
\end{verbatim}

\begin{verbatim}
## The following object is masked from 'package:Metrics':
## 
##     auc
\end{verbatim}

\begin{Shaded}
\begin{Highlighting}[]
\KeywordTok{library}\NormalTok{(lars); }\KeywordTok{library}\NormalTok{(MASS);}\KeywordTok{library}\NormalTok{(glmnet)}
\end{Highlighting}
\end{Shaded}

\begin{verbatim}
## Loaded lars 1.2
\end{verbatim}

\begin{verbatim}
## Loading required package: Matrix
\end{verbatim}

\begin{verbatim}
## Loaded glmnet 4.0-2
\end{verbatim}

\hypertarget{lecture-des-donnuxe9es}{%
\subsection{Lecture des données}\label{lecture-des-donnuxe9es}}

On sépare les prix en 2 classes 0 et 1. Une transaction appartient à la
classe 1 si son prix est supérieure à la médiane de la variable prix et
0 sinon. Préduire si une transaction appartient à une de ces 2 classes
peut se faire par une régressions logistiques.

\begin{Shaded}
\begin{Highlighting}[]
\NormalTok{tab=}\KeywordTok{read.table}\NormalTok{(}\StringTok{"RealEstate.csv"}\NormalTok{,}\DataTypeTok{header=}\OtherTok{TRUE}\NormalTok{,}\DataTypeTok{sep=}\StringTok{','}\NormalTok{);}
\NormalTok{medianHousePrice=}\KeywordTok{median}\NormalTok{(tab}\OperatorTok{$}\NormalTok{Y.house.price.of.unit.area);}
\NormalTok{medHousePriceBin=}\KeywordTok{as.numeric}\NormalTok{(tab}\OperatorTok{$}\NormalTok{Y.house.price.of.unit.area}\OperatorTok{>}\NormalTok{medianHousePrice);}

\CommentTok{## }
\NormalTok{tabmed =}\StringTok{ }\NormalTok{tab}
\NormalTok{tabmed =}\StringTok{ }\NormalTok{tabmed[,}\OperatorTok{-}\DecValTok{1}\NormalTok{]}
\NormalTok{tabmed}\OperatorTok{$}\NormalTok{Y.house.price.of.unit.area =}\StringTok{ }\NormalTok{medHousePriceBin}
\KeywordTok{head}\NormalTok{(tabmed)}
\end{Highlighting}
\end{Shaded}

\begin{verbatim}
##   X1.transaction.date X2.house.age X3.distance.to.the.nearest.MRT.station
## 1            2012.917         32.0                               84.87882
## 2            2012.917         19.5                              306.59470
## 3            2013.583         13.3                              561.98450
## 4            2013.500         13.3                              561.98450
## 5            2012.833          5.0                              390.56840
## 6            2012.667          7.1                             2175.03000
##   X4.number.of.convenience.stores X5.latitude X6.longitude
## 1                              10    24.98298     121.5402
## 2                               9    24.98034     121.5395
## 3                               5    24.98746     121.5439
## 4                               5    24.98746     121.5439
## 5                               5    24.97937     121.5425
## 6                               3    24.96305     121.5125
##   Y.house.price.of.unit.area
## 1                          0
## 2                          1
## 3                          1
## 4                          1
## 5                          1
## 6                          0
\end{verbatim}

\begin{Shaded}
\begin{Highlighting}[]
\KeywordTok{colnames}\NormalTok{(tabmed)[}\KeywordTok{dim}\NormalTok{(tabmed)[}\DecValTok{2}\NormalTok{]] <-}\StringTok{ "medHousePriceBin"} \CommentTok{# change la variable price en medHousePriceBin dans le tableau.}
\KeywordTok{head}\NormalTok{(tabmed)}
\end{Highlighting}
\end{Shaded}

\begin{verbatim}
##   X1.transaction.date X2.house.age X3.distance.to.the.nearest.MRT.station
## 1            2012.917         32.0                               84.87882
## 2            2012.917         19.5                              306.59470
## 3            2013.583         13.3                              561.98450
## 4            2013.500         13.3                              561.98450
## 5            2012.833          5.0                              390.56840
## 6            2012.667          7.1                             2175.03000
##   X4.number.of.convenience.stores X5.latitude X6.longitude medHousePriceBin
## 1                              10    24.98298     121.5402                0
## 2                               9    24.98034     121.5395                1
## 3                               5    24.98746     121.5439                1
## 4                               5    24.98746     121.5439                1
## 5                               5    24.97937     121.5425                1
## 6                               3    24.96305     121.5125                0
\end{verbatim}

\hypertarget{visualisation-des-donnuxe9es}{%
\subsection{visualisation des
données}\label{visualisation-des-donnuxe9es}}

Le plot des différentes covariables nous permet de distinguer les
dependances a priori entre elles.

\begin{Shaded}
\begin{Highlighting}[]
\NormalTok{mcor =}\StringTok{ }\KeywordTok{cor}\NormalTok{(tabmed) }\CommentTok{# correlation matrix}
\KeywordTok{corrplot}\NormalTok{(mcor, }\DataTypeTok{method=}\StringTok{"color"}\NormalTok{, }\DataTypeTok{addCoef.col=} \StringTok{"black"}\NormalTok{, }\DataTypeTok{tl.srt =}
\DecValTok{45}\NormalTok{, }\DataTypeTok{sig.level=}\FloatTok{0.01}\NormalTok{, }\DataTypeTok{insig=}\StringTok{"blank"}\NormalTok{)}
\end{Highlighting}
\end{Shaded}

\includegraphics{TP3_files/figure-latex/unnamed-chunk-3-1.pdf}

\begin{Shaded}
\begin{Highlighting}[]
\KeywordTok{pairs}\NormalTok{(tabmed,}\DataTypeTok{pch=}\DecValTok{22}\NormalTok{,}\DataTypeTok{bg=}\KeywordTok{c}\NormalTok{(}\StringTok{"red"}\NormalTok{,}\StringTok{"blue"}\NormalTok{)[}\KeywordTok{unclass}\NormalTok{(}\KeywordTok{factor}\NormalTok{(tabmed[,}\StringTok{"medHousePriceBin"}\NormalTok{]))]) }\CommentTok{# }
\end{Highlighting}
\end{Shaded}

\includegraphics{TP3_files/figure-latex/unnamed-chunk-3-2.pdf}

\begin{Shaded}
\begin{Highlighting}[]
\CommentTok{#dans le plot les points rouges sont les transactions dont le prix sont}
\CommentTok{# en dessous de la mediane [0], et le bleus celles au dessus [1]}
\end{Highlighting}
\end{Shaded}

Dans ce graphe les points bleus sont les transactions dont le prix est
supérieur à la médiane et les rouges sont celles dont le prix est
inférieur. A part, X1.transaction.date/X2.house.date, sur lequel on peut
pas distinguer des clusters, sur les autres plots , on voit nettement
des clusterS se former dans chacun des covariables plots. La proximité à
la station la plus proche X3 est une variables fortement corrélée avec
la X6.longitude.

\hypertarget{i-les-moduxe8les.}{%
\section{I) Les modèles.}\label{i-les-moduxe8les.}}

Dans la suite on va essayer de generer differents modèles à partir des
regression logistique sur notre eu de données:

On commencera par un modèle dit ``full'', qui utiliseras toutes les
variables, ensuite on fera une selection de variables grace une méthode
``stepwise'', puis on fera des regressions logistiques sous contraintes
classiques en utilisant ``ridge'' et ``lasso''.

\hypertarget{organisation-du-dataset}{%
\subsubsection{organisation du dataset}\label{organisation-du-dataset}}

on va partitionner notre jeu de données en training et test avec des
proportions respectives de 80\% et 20\%.

\begin{Shaded}
\begin{Highlighting}[]
\KeywordTok{set.seed}\NormalTok{(}\DecValTok{1234}\NormalTok{) }\CommentTok{#assure la repetabilité des differents pocessus aleatoire}
\NormalTok{p =}\StringTok{ }\FloatTok{0.8} \CommentTok{# represente 80% qui va  etre pour le trainin set }
\NormalTok{ind =}\StringTok{ }\KeywordTok{sample}\NormalTok{(}\DecValTok{2}\NormalTok{, }\KeywordTok{nrow}\NormalTok{(tabmed), }\DataTypeTok{replace =}\NormalTok{ T, }\DataTypeTok{prob =} \KeywordTok{c}\NormalTok{(p,}\DecValTok{1}\OperatorTok{-}\NormalTok{p)) }\CommentTok{## selectionne les indices}
\CommentTok{#aleatoirent entre 2 groupe, le premiere groupe represente 80 % de l'ensemble }
\CommentTok{#des indices et le 2e groupe le reste.}
\NormalTok{tab.train =}\StringTok{ }\KeywordTok{as.data.frame}\NormalTok{(tabmed[ind }\OperatorTok{==}\StringTok{ }\DecValTok{1}\NormalTok{,]) }\CommentTok{#training dataset}
\NormalTok{tab.test =}\StringTok{ }\KeywordTok{as.data.frame}\NormalTok{(tabmed[ind }\OperatorTok{==}\StringTok{ }\DecValTok{2}\NormalTok{,]) }\CommentTok{#test dataset}

\NormalTok{X.train =}\StringTok{ }\KeywordTok{as.matrix}\NormalTok{(tab.train[,}\OperatorTok{-}\KeywordTok{dim}\NormalTok{(tab.train)[}\DecValTok{2}\NormalTok{]])}
\NormalTok{X.test =}\StringTok{ }\KeywordTok{as.matrix}\NormalTok{(tab.test[,}\OperatorTok{-}\KeywordTok{dim}\NormalTok{(tab.test)[}\DecValTok{2}\NormalTok{]])}
\NormalTok{Y.test =}\StringTok{ }\NormalTok{tab.test}\OperatorTok{$}\NormalTok{medHousePriceBin}
\NormalTok{Y.train =}\StringTok{ }\NormalTok{tab.train}\OperatorTok{$}\NormalTok{medHousePriceBin}
\end{Highlighting}
\end{Shaded}

\hypertarget{i.1-modele-full}{%
\subsection{I.1 Modele full}\label{i.1-modele-full}}

\hypertarget{a-model-full}{%
\subsubsection{a) model full}\label{a-model-full}}

\begin{Shaded}
\begin{Highlighting}[]
\NormalTok{model.full =}\StringTok{ }\KeywordTok{glm}\NormalTok{(medHousePriceBin }\OperatorTok{~}\StringTok{ }\NormalTok{., }\DataTypeTok{data =}\NormalTok{ tab.train, }\DataTypeTok{family =} \StringTok{'binomial'}\NormalTok{)}
\KeywordTok{summary}\NormalTok{(model.full)}
\end{Highlighting}
\end{Shaded}

\begin{verbatim}
## 
## Call:
## glm(formula = medHousePriceBin ~ ., family = "binomial", data = tab.train)
## 
## Deviance Residuals: 
##     Min       1Q   Median       3Q      Max  
## -2.3966  -0.3026   0.1507   0.5368   3.1611  
## 
## Coefficients:
##                                          Estimate Std. Error z value Pr(>|z|)
## (Intercept)                            -2.404e+03  3.908e+03  -0.615  0.53834
## X1.transaction.date                     1.381e+00  6.299e-01   2.193  0.02834
## X2.house.age                           -7.016e-02  1.460e-02  -4.807 1.53e-06
## X3.distance.to.the.nearest.MRT.station -2.596e-03  5.141e-04  -5.049 4.44e-07
## X4.number.of.convenience.stores         2.619e-01  8.212e-02   3.189  0.00143
## X5.latitude                             9.477e+01  2.235e+01   4.241 2.22e-05
## X6.longitude                           -2.255e+01  3.117e+01  -0.723  0.46946
##                                           
## (Intercept)                               
## X1.transaction.date                    *  
## X2.house.age                           ***
## X3.distance.to.the.nearest.MRT.station ***
## X4.number.of.convenience.stores        ** 
## X5.latitude                            ***
## X6.longitude                              
## ---
## Signif. codes:  0 '***' 0.001 '**' 0.01 '*' 0.05 '.' 0.1 ' ' 1
## 
## (Dispersion parameter for binomial family taken to be 1)
## 
##     Null deviance: 467.11  on 336  degrees of freedom
## Residual deviance: 221.48  on 330  degrees of freedom
## AIC: 235.48
## 
## Number of Fisher Scoring iterations: 7
\end{verbatim}

Avec un seuil de p-value à 0.01, les variables statistiquement
significatives sont X2,X3,X4 et X5. De plus le test rejete la varible
X6.longitude. Ce qui est sans doute du à la corrélation de celle ci avec
X3.

\hypertarget{b-prediction-avec-le-model-full}{%
\subsubsection{b) Prediction avec le model
full}\label{b-prediction-avec-le-model-full}}

\begin{Shaded}
\begin{Highlighting}[]
 \CommentTok{### prediction}
\NormalTok{prob =}\StringTok{ }\KeywordTok{predict.glm}\NormalTok{(model.full, }\DataTypeTok{newdata =}\NormalTok{ tab.test,}\DataTypeTok{type =} \StringTok{"response"}\NormalTok{) }\CommentTok{# give the predicted probability}
\NormalTok{OR_full =}\StringTok{ }\KeywordTok{exp}\NormalTok{(model.full}\OperatorTok{$}\NormalTok{coefficients) }\CommentTok{# odd ratio}
\NormalTok{OR_full}
\end{Highlighting}
\end{Shaded}

\begin{verbatim}
##                            (Intercept)                    X1.transaction.date 
##                           0.000000e+00                           3.979501e+00 
##                           X2.house.age X3.distance.to.the.nearest.MRT.station 
##                           9.322441e-01                           9.974075e-01 
##        X4.number.of.convenience.stores                            X5.latitude 
##                           1.299354e+00                           1.439174e+41 
##                           X6.longitude 
##                           1.609538e-10
\end{verbatim}

Une augmentation de X4 le nombre de magasins de proximité, et la
latitute X5 ou la date de transaction X1 semble augmenter augmenter le
prix de la transaction. Tandis que une augmententation des autres
variables telles que le l'age de maison X2, la longitude X6 ou la
distance a la station la plus proche X3 semble diminuer le prix de la
transaction.

\hypertarget{c-performance-du-model-full}{%
\subsubsection{c) Performance du model
full}\label{c-performance-du-model-full}}

\begin{Shaded}
\begin{Highlighting}[]
\NormalTok{Threshold =}\StringTok{ }\FloatTok{0.5} \CommentTok{# for MAE estimation}
\NormalTok{Y.pred.full =}\StringTok{  }\KeywordTok{as.integer}\NormalTok{(prob }\OperatorTok{>=}\StringTok{ }\NormalTok{Threshold) }
\NormalTok{confusion_matrix.full =}\StringTok{ }\KeywordTok{table}\NormalTok{(Y.pred.full,tab.test}\OperatorTok{$}\NormalTok{medHousePriceBin)}
\NormalTok{confusion_matrix.full}
\end{Highlighting}
\end{Shaded}

\begin{verbatim}
##            
## Y.pred.full  0  1
##           0 33  9
##           1  8 27
\end{verbatim}

La matrice de confusion nous donne une performance

\begin{Shaded}
\begin{Highlighting}[]
\NormalTok{accuracy.full =}\StringTok{ }\KeywordTok{sum}\NormalTok{(}\KeywordTok{diag}\NormalTok{(confusion_matrix.full))}\OperatorTok{/}\KeywordTok{sum}\NormalTok{(confusion_matrix.full)}\OperatorTok{*}\DecValTok{100}\CommentTok{#   }
\NormalTok{  recall.full =}\StringTok{ }\NormalTok{confusion_matrix.full[}\DecValTok{2}\NormalTok{,}\DecValTok{2}\NormalTok{]}\OperatorTok{/}\KeywordTok{sum}\NormalTok{(confusion_matrix.full[,}\DecValTok{2}\NormalTok{])}\OperatorTok{*}\DecValTok{100} \CommentTok{#}
\NormalTok{  precision.full =confusion_matrix.full[}\DecValTok{2}\NormalTok{,}\DecValTok{2}\NormalTok{]}\OperatorTok{/}\KeywordTok{sum}\NormalTok{(confusion_matrix.full[}\DecValTok{2}\NormalTok{,])}\OperatorTok{*}\DecValTok{100}
\NormalTok{  error_rate.full =}\StringTok{ }\KeywordTok{sum}\NormalTok{(}\KeywordTok{diag}\NormalTok{(confusion_matrix.full[}\DecValTok{1}\OperatorTok{:}\DecValTok{2}\NormalTok{,}\DecValTok{2}\OperatorTok{:}\DecValTok{1}\NormalTok{]))}\OperatorTok{/}\KeywordTok{sum}\NormalTok{(confusion_matrix.full) }\OperatorTok{*}\DecValTok{100} 
\end{Highlighting}
\end{Shaded}

Avec une specificité de 80\% et un un recall de 75\% le model full
prédit assez bien les éléments de differentes classe. Globalement on
fait une erreur de 22\%, ce qui relativement bas au vu de la taille de
l'echantillon.

Afin d'évaluer d'avoir une meilleur fiabilité à notre modèle on va
effecteur une Validation croisée.

\hypertarget{d-k-folds-pour-le-model-full}{%
\subsubsection{d) k-folds pour le model
full}\label{d-k-folds-pour-le-model-full}}

\begin{Shaded}
\begin{Highlighting}[]
\CommentTok{##shuffling}
\KeywordTok{set.seed}\NormalTok{(}\DecValTok{1234}\NormalTok{)}
\NormalTok{rows <-}\StringTok{ }\KeywordTok{sample}\NormalTok{(}\KeywordTok{nrow}\NormalTok{(tabmed)) }\CommentTok{# melange des rows du dataset}
\NormalTok{tabmed <-}\StringTok{ }\NormalTok{tabmed[rows, ]}
\CommentTok{## folds}
\NormalTok{k =}\StringTok{ }\DecValTok{15} \CommentTok{#as.integer(1/(1-r)) ## fold number}
\NormalTok{fold =}\StringTok{ }\KeywordTok{cut}\NormalTok{(}\KeywordTok{seq}\NormalTok{(}\DecValTok{1}\NormalTok{,}\KeywordTok{nrow}\NormalTok{(tabmed)), }\DataTypeTok{breaks =}\NormalTok{ k,}\DataTypeTok{labels =} \OtherTok{FALSE}\NormalTok{) }\CommentTok{#organisations des lignes du}
\CommentTok{#dataset en plusieurs fold}
\CommentTok{#initialisation des vectueurs qui va contenier les performances a chaque etape}
\NormalTok{accuracyk.full =}\StringTok{ }\KeywordTok{c}\NormalTok{()}
\NormalTok{recallk.full =}\StringTok{ }\KeywordTok{c}\NormalTok{()}
\NormalTok{precisionk.full =}\StringTok{ }\KeywordTok{c}\NormalTok{()}
\NormalTok{error_ratek.full =}\StringTok{ }\KeywordTok{c}\NormalTok{()}

\ControlFlowTok{for}\NormalTok{ (i }\ControlFlowTok{in} \DecValTok{1}\OperatorTok{:}\NormalTok{k) \{}
\NormalTok{  test_rows =}\StringTok{ }\KeywordTok{which}\NormalTok{(fold }\OperatorTok{==}\StringTok{ }\NormalTok{i,}\DataTypeTok{arr.ind =} \OtherTok{TRUE}\NormalTok{) }\CommentTok{#selection des lignes qui correspondent }
  \CommentTok{# au fold i}
\NormalTok{  tab.testk =}\StringTok{ }\NormalTok{tabmed[test_rows,]}
\NormalTok{  tab.traink =}\StringTok{ }\NormalTok{tabmed[}\OperatorTok{-}\NormalTok{test_rows,]}
  \CommentTok{### regression logistic }
\NormalTok{  model.fullk=}\KeywordTok{glm}\NormalTok{(medHousePriceBin}\OperatorTok{~}\NormalTok{.,}\DataTypeTok{family=}\NormalTok{binomial,}\DataTypeTok{data =}\NormalTok{ tab.traink)}
  \CommentTok{### prediction}
\NormalTok{  prob =}\StringTok{ }\KeywordTok{predict.glm}\NormalTok{(model.fullk, }\DataTypeTok{newdata =}\NormalTok{ tab.testk,}\DataTypeTok{type =} \StringTok{"response"}\NormalTok{) }\CommentTok{# give probability}
\NormalTok{  Y.pred.full =}\StringTok{  }\KeywordTok{as.integer}\NormalTok{(prob }\OperatorTok{>=}\StringTok{ }\NormalTok{Threshold)  }\CommentTok{#MAE estimation}
\NormalTok{  confusion_matrix =}\StringTok{ }\KeywordTok{table}\NormalTok{(Y.pred.full,tab.testk}\OperatorTok{$}\NormalTok{medHousePriceBin)}
\NormalTok{  accuracyk.full[i] =}\StringTok{ }\KeywordTok{sum}\NormalTok{(}\KeywordTok{diag}\NormalTok{(confusion_matrix))}\OperatorTok{/}\KeywordTok{sum}\NormalTok{(confusion_matrix)}\OperatorTok{*}\DecValTok{100}\CommentTok{#   }
\NormalTok{  recallk.full[i] =}\StringTok{ }\NormalTok{confusion_matrix[}\DecValTok{2}\NormalTok{,}\DecValTok{2}\NormalTok{]}\OperatorTok{/}\KeywordTok{sum}\NormalTok{(confusion_matrix[,}\DecValTok{2}\NormalTok{])}\OperatorTok{*}\DecValTok{100} \CommentTok{#}
\NormalTok{  precisionk.full[i] =confusion_matrix[}\DecValTok{2}\NormalTok{,}\DecValTok{2}\NormalTok{]}\OperatorTok{/}\KeywordTok{sum}\NormalTok{(confusion_matrix[}\DecValTok{2}\NormalTok{,])}\OperatorTok{*}\DecValTok{100}
\NormalTok{  error_ratek.full[i] =}\StringTok{ }\KeywordTok{sum}\NormalTok{(}\KeywordTok{diag}\NormalTok{(confusion_matrix[}\DecValTok{1}\OperatorTok{:}\DecValTok{2}\NormalTok{,}\DecValTok{2}\OperatorTok{:}\DecValTok{1}\NormalTok{]))}\OperatorTok{/}\KeywordTok{sum}\NormalTok{(confusion_matrix) }\OperatorTok{*}\DecValTok{100} 
\NormalTok{\}}
\KeywordTok{mean}\NormalTok{(recallk.full)}
\end{Highlighting}
\end{Shaded}

\begin{verbatim}
## [1] 86.26258
\end{verbatim}

\begin{Shaded}
\begin{Highlighting}[]
\KeywordTok{mean}\NormalTok{(error_ratek.full)}
\end{Highlighting}
\end{Shaded}

\begin{verbatim}
## [1] 18.11287
\end{verbatim}

\begin{Shaded}
\begin{Highlighting}[]
\KeywordTok{mean}\NormalTok{(accuracyk.full)}
\end{Highlighting}
\end{Shaded}

\begin{verbatim}
## [1] 81.88713
\end{verbatim}

\begin{Shaded}
\begin{Highlighting}[]
\KeywordTok{mean}\NormalTok{(precisionk.full)}
\end{Highlighting}
\end{Shaded}

\begin{verbatim}
## [1] 78.74136
\end{verbatim}

En utilisant le k-fold on évalue la precision du model. nous donne une
bonne performance global. Avec une performance global de 80\% et un taux
d'erreur d'environ 20\%.

\hypertarget{i.2-regression-logistique-avec-selection-de-variable-stepwise}{%
\subsection{I.2) Regression Logistique avec selection de variable
Stepwise}\label{i.2-regression-logistique-avec-selection-de-variable-stepwise}}

On a vu que certaines variables, comme X6.longitude était pas
significative dans notre model full,dans cette partie, on va effectuer
une selection de Variables, puis une regression logistique sur ces
dernières.

\hypertarget{a-resstep-model}{%
\subsubsection{a) Resstep Model}\label{a-resstep-model}}

Selection des variables à partir du model précédent.

\begin{Shaded}
\begin{Highlighting}[]
\NormalTok{resstep<-}\KeywordTok{step}\NormalTok{(model.full,}\DataTypeTok{direction=}\StringTok{'both'}\NormalTok{); }\KeywordTok{summary}\NormalTok{(resstep)}
\end{Highlighting}
\end{Shaded}

\begin{verbatim}
## Start:  AIC=235.48
## medHousePriceBin ~ X1.transaction.date + X2.house.age + X3.distance.to.the.nearest.MRT.station + 
##     X4.number.of.convenience.stores + X5.latitude + X6.longitude
## 
##                                          Df Deviance    AIC
## - X6.longitude                            1   222.02 234.02
## <none>                                        221.48 235.48
## - X1.transaction.date                     1   226.41 238.41
## - X4.number.of.convenience.stores         1   232.81 244.81
## - X5.latitude                             1   242.44 254.44
## - X2.house.age                            1   248.18 260.18
## - X3.distance.to.the.nearest.MRT.station  1   265.27 277.27
## 
## Step:  AIC=234.02
## medHousePriceBin ~ X1.transaction.date + X2.house.age + X3.distance.to.the.nearest.MRT.station + 
##     X4.number.of.convenience.stores + X5.latitude
## 
##                                          Df Deviance    AIC
## <none>                                        222.02 234.02
## + X6.longitude                            1   221.48 235.48
## - X1.transaction.date                     1   226.74 236.74
## - X4.number.of.convenience.stores         1   232.94 242.94
## - X5.latitude                             1   243.59 253.59
## - X2.house.age                            1   249.18 259.18
## - X3.distance.to.the.nearest.MRT.station  1   279.21 289.21
\end{verbatim}

\begin{verbatim}
## 
## Call:
## glm(formula = medHousePriceBin ~ X1.transaction.date + X2.house.age + 
##     X3.distance.to.the.nearest.MRT.station + X4.number.of.convenience.stores + 
##     X5.latitude, family = "binomial", data = tab.train)
## 
## Deviance Residuals: 
##     Min       1Q   Median       3Q      Max  
## -2.3391  -0.2916   0.1624   0.5223   3.1496  
## 
## Coefficients:
##                                          Estimate Std. Error z value Pr(>|z|)
## (Intercept)                            -5.072e+03  1.404e+03  -3.612 0.000304
## X1.transaction.date                     1.348e+00  6.286e-01   2.145 0.031950
## X2.house.age                           -7.077e-02  1.461e-02  -4.843 1.28e-06
## X3.distance.to.the.nearest.MRT.station -2.417e-03  4.475e-04  -5.400 6.66e-08
## X4.number.of.convenience.stores         2.392e-01  7.401e-02   3.232 0.001231
## X5.latitude                             9.448e+01  2.205e+01   4.284 1.83e-05
##                                           
## (Intercept)                            ***
## X1.transaction.date                    *  
## X2.house.age                           ***
## X3.distance.to.the.nearest.MRT.station ***
## X4.number.of.convenience.stores        ** 
## X5.latitude                            ***
## ---
## Signif. codes:  0 '***' 0.001 '**' 0.01 '*' 0.05 '.' 0.1 ' ' 1
## 
## (Dispersion parameter for binomial family taken to be 1)
## 
##     Null deviance: 467.11  on 336  degrees of freedom
## Residual deviance: 222.02  on 331  degrees of freedom
## AIC: 234.02
## 
## Number of Fisher Scoring iterations: 7
\end{verbatim}

Sans surprise, comme vu dans le TP2, la variable X6.longitude a été
éliminée dans la sélection de variables, ce qui augmente la
significaivité des autres variables, c'est ce qu'on remarque en
comparant les test effectués sur une régression logistique de type full
(toutes les variables) et les test sur une régression logistique de type
stepwise.

\begin{Shaded}
\begin{Highlighting}[]
\NormalTok{OR_step=}\KeywordTok{exp}\NormalTok{(resstep}\OperatorTok{$}\NormalTok{coefficients) }\CommentTok{# give the odd ratios on stepwise model}
\NormalTok{OR_step}
\end{Highlighting}
\end{Shaded}

\begin{verbatim}
##                            (Intercept)                    X1.transaction.date 
##                           0.000000e+00                           3.850789e+00 
##                           X2.house.age X3.distance.to.the.nearest.MRT.station 
##                           9.316744e-01                           9.975862e-01 
##        X4.number.of.convenience.stores                            X5.latitude 
##                           1.270203e+00                           1.074748e+41
\end{verbatim}

Comme precedemment, les variables X1.transaction date,
X4.convenience.stores et X5.latitude, leur augmentation semble entrainer
une augmentation de la probabilité que medHousePriceBin soit au-dessus
de la médiane du prix de vente, au risque de 5\%.

\hypertarget{b-pruxe9diction-sur-le-model-stepwise}{%
\subsection{b ) Prédiction sur le model
stepwise}\label{b-pruxe9diction-sur-le-model-stepwise}}

\begin{Shaded}
\begin{Highlighting}[]
\NormalTok{prob=}\StringTok{ }\KeywordTok{predict.glm}\NormalTok{(resstep, }\DataTypeTok{newdata =}\NormalTok{ tab.test,}\DataTypeTok{type =} \StringTok{"response"}\NormalTok{) }\CommentTok{# give the predicted prob on the stepwise model}
\NormalTok{pred_step=}\KeywordTok{as.numeric}\NormalTok{(prob}\OperatorTok{>}\NormalTok{Threshold) }\CommentTok{#}
\end{Highlighting}
\end{Shaded}

On calcule les prédictions contruites à partir de ce modèle pred\_step
qui donne la valeur des \(\hat\eta\) pour tous les individus dans le
data set test. pred\_step vaut 1 si pred\_step \textgreater{} 1/2.

\begin{Shaded}
\begin{Highlighting}[]
\NormalTok{confusion_matrix.step =}\StringTok{ }\KeywordTok{table}\NormalTok{(pred_step,tab.test}\OperatorTok{$}\NormalTok{medHousePriceBin)}
\NormalTok{confusion_matrix.step}
\end{Highlighting}
\end{Shaded}

\begin{verbatim}
##          
## pred_step  0  1
##         0 33  7
##         1  8 29
\end{verbatim}

La matrice de confusion nous montre qu'on predit mieux de vrai
evenements qu'on en fait des erreurs.

\begin{Shaded}
\begin{Highlighting}[]
\NormalTok{pred.accuracy.step =}\StringTok{ }\KeywordTok{sum}\NormalTok{(}\KeywordTok{diag}\NormalTok{(confusion_matrix.step))}\OperatorTok{/}\KeywordTok{sum}\NormalTok{(confusion_matrix.step)}\OperatorTok{*}\DecValTok{100}
\CommentTok{#prediction accuracy}
\NormalTok{pred.recall.step =}\StringTok{ }\NormalTok{confusion_matrix.step[}\DecValTok{2}\NormalTok{,}\DecValTok{2}\NormalTok{]}\OperatorTok{/}\KeywordTok{sum}\NormalTok{(confusion_matrix.step[,}\DecValTok{2}\NormalTok{])}\OperatorTok{*}\DecValTok{100} 
\CommentTok{# probabilité que la prediction des hauts prix soit correctes }
\NormalTok{pred.specifity.step =}\StringTok{ }\NormalTok{confusion_matrix.step[}\DecValTok{1}\NormalTok{,}\DecValTok{1}\NormalTok{]}\OperatorTok{/}\KeywordTok{sum}\NormalTok{(confusion_matrix.step[,}\DecValTok{1}\NormalTok{])}\OperatorTok{*}\DecValTok{100} 
\CommentTok{# probabilité que la prediction des bas prix soit correctes }
\NormalTok{pred.precision.step =}\StringTok{ }\NormalTok{confusion_matrix.step[}\DecValTok{2}\NormalTok{,}\DecValTok{2}\NormalTok{]}\OperatorTok{/}\KeywordTok{sum}\NormalTok{(confusion_matrix.step[}\DecValTok{2}\NormalTok{,])}\OperatorTok{*}\DecValTok{100}
\CommentTok{# }
\NormalTok{pred.error_rate.step =}\StringTok{ }\KeywordTok{sum}\NormalTok{(}\KeywordTok{diag}\NormalTok{(confusion_matrix.step[}\DecValTok{1}\OperatorTok{:}\DecValTok{2}\NormalTok{,}\DecValTok{2}\OperatorTok{:}\DecValTok{1}\NormalTok{]))}\OperatorTok{/}\KeywordTok{sum}\NormalTok{(confusion_matrix.step) }\OperatorTok{*}\DecValTok{100} \CommentTok{# probabilité d'obtenir une erreur}

\NormalTok{pred.accuracy.step}
\end{Highlighting}
\end{Shaded}

\begin{verbatim}
## [1] 80.51948
\end{verbatim}

\begin{Shaded}
\begin{Highlighting}[]
\NormalTok{pred.recall.step}
\end{Highlighting}
\end{Shaded}

\begin{verbatim}
## [1] 80.55556
\end{verbatim}

\begin{Shaded}
\begin{Highlighting}[]
\NormalTok{pred.specifity.step}
\end{Highlighting}
\end{Shaded}

\begin{verbatim}
## [1] 80.4878
\end{verbatim}

\begin{Shaded}
\begin{Highlighting}[]
\NormalTok{pred.error_rate.step}
\end{Highlighting}
\end{Shaded}

\begin{verbatim}
## [1] 19.48052
\end{verbatim}

Comme dit precedemment mais avec des chiffres, on effectue des bonnes
prédictions en moyenne à 80.5\%. Ce qui est pas un bon score au vu de la
taille de l'échantillon.

\hypertarget{courbe-roc-et-auc}{%
\subsubsection{Courbe ROC et AUC}\label{courbe-roc-et-auc}}

\begin{Shaded}
\begin{Highlighting}[]
\KeywordTok{library}\NormalTok{(ROCR)}

\NormalTok{pred=}\KeywordTok{prediction}\NormalTok{(pred_step,tab.test}\OperatorTok{$}\NormalTok{medHousePriceBin)}
\NormalTok{perf=}\KeywordTok{performance}\NormalTok{(pred, }\StringTok{"tpr"}\NormalTok{ ,}\StringTok{"fpr"}\NormalTok{)}
\KeywordTok{plot}\NormalTok{(perf)}
\end{Highlighting}
\end{Shaded}

\includegraphics{TP3_files/figure-latex/unnamed-chunk-15-1.pdf}

La courbe est assez proche du coin supérieur gauche, ainsi le modèle
semble capturer le plus possible de vrais évènements avec le moins
possible de faux évènements.

\begin{Shaded}
\begin{Highlighting}[]
\NormalTok{ROC_auc=}\KeywordTok{performance}\NormalTok{(pred,}\StringTok{"auc"}\NormalTok{)}
\NormalTok{AUC=ROC_auc}\OperatorTok{@}\NormalTok{y.values[[}\DecValTok{1}\NormalTok{]]}
\KeywordTok{print}\NormalTok{(AUC)}
\end{Highlighting}
\end{Shaded}

\begin{verbatim}
## [1] 0.8052168
\end{verbatim}

L'AUC est de 0.805.

\hypertarget{c-k-fold-pour-le-model-stepwise}{%
\subsubsection{c) k-fold pour le model
stepwise}\label{c-k-fold-pour-le-model-stepwise}}

On effectue 15 folds

\begin{Shaded}
\begin{Highlighting}[]
\NormalTok{rows=}\KeywordTok{sample}\NormalTok{(}\KeywordTok{nrow}\NormalTok{(tabmed))}
\NormalTok{tabmedk=tabmed[rows,]}

\CommentTok{## folds}

\NormalTok{k=}\DecValTok{15}
\NormalTok{fold=}\KeywordTok{cut}\NormalTok{(}\KeywordTok{seq}\NormalTok{(}\DecValTok{1}\NormalTok{,}\KeywordTok{nrow}\NormalTok{(tabmed)),}\DataTypeTok{breaks=}\NormalTok{k, }\DataTypeTok{labels=}\OtherTok{FALSE}\NormalTok{)}
\CommentTok{##}
\NormalTok{accuracyk.step=}\KeywordTok{c}\NormalTok{()}
\NormalTok{recallk.step=}\KeywordTok{c}\NormalTok{()}
\NormalTok{error_ratek.step=}\KeywordTok{c}\NormalTok{()}
\NormalTok{precisionk.step=}\KeywordTok{c}\NormalTok{()}

\ControlFlowTok{for}\NormalTok{ (i }\ControlFlowTok{in} \DecValTok{1}\OperatorTok{:}\NormalTok{k)\{}
\NormalTok{  test_rows=}\KeywordTok{which}\NormalTok{(fold}\OperatorTok{==}\NormalTok{i, }\DataTypeTok{arr.ind=}\OtherTok{TRUE}\NormalTok{)}
\NormalTok{  tab.testk=tabmedk[test_rows,]}
\NormalTok{  tab.traink=tabmedk[}\OperatorTok{-}\NormalTok{test_rows,]}
  
  \CommentTok{# regression logistique}
\NormalTok{  model.full=}\KeywordTok{glm}\NormalTok{(medHousePriceBin}\OperatorTok{~}\NormalTok{., }\DataTypeTok{family=}\NormalTok{binomial, }\DataTypeTok{data=}\NormalTok{tab.traink)}
\NormalTok{  model.stepk=}\KeywordTok{step}\NormalTok{(model.full,}\DataTypeTok{direction=}\StringTok{'both'}\NormalTok{)}
  
  \CommentTok{# prediction}
  
\NormalTok{  pred_stepk=}\KeywordTok{predict.glm}\NormalTok{(model.stepk, }\DataTypeTok{newdata=}\NormalTok{ tab.testk, }\DataTypeTok{type=}\StringTok{"response"}\NormalTok{)}
\NormalTok{  Y.pred.step=}\KeywordTok{as.numeric}\NormalTok{(pred_stepk}\OperatorTok{>=}\NormalTok{Threshold)}
\NormalTok{  confusion_matrix=}\KeywordTok{table}\NormalTok{(Y.pred.step, tab.testk}\OperatorTok{$}\NormalTok{medHousePriceBin)}
\NormalTok{  accuracyk.step[i]=}\KeywordTok{sum}\NormalTok{(}\KeywordTok{diag}\NormalTok{(confusion_matrix))}\OperatorTok{/}\KeywordTok{sum}\NormalTok{(confusion_matrix)}\OperatorTok{*}\DecValTok{100} \CommentTok{# prediction accuracy }
\NormalTok{  precisionk.step[i] =confusion_matrix[}\DecValTok{2}\NormalTok{,}\DecValTok{2}\NormalTok{]}\OperatorTok{/}\KeywordTok{sum}\NormalTok{(confusion_matrix[}\DecValTok{2}\NormalTok{,])}\OperatorTok{*}\DecValTok{100}
\NormalTok{  recallk.step[i]=confusion_matrix[}\DecValTok{2}\NormalTok{,}\DecValTok{2}\NormalTok{]}\OperatorTok{/}\KeywordTok{sum}\NormalTok{(confusion_matrix[,}\DecValTok{2}\NormalTok{])}\OperatorTok{*}\DecValTok{100} 
\NormalTok{  error_ratek.step[i]=}\KeywordTok{sum}\NormalTok{(}\KeywordTok{diag}\NormalTok{(confusion_matrix[}\DecValTok{1}\OperatorTok{:}\DecValTok{2}\NormalTok{,}\DecValTok{2}\OperatorTok{:}\DecValTok{1}\NormalTok{]))}\OperatorTok{/}\KeywordTok{sum}\NormalTok{(confusion_matrix)}\OperatorTok{*}\DecValTok{100}
\NormalTok{\}}
\end{Highlighting}
\end{Shaded}

\begin{verbatim}
## Start:  AIC=290.05
## medHousePriceBin ~ X1.transaction.date + X2.house.age + X3.distance.to.the.nearest.MRT.station + 
##     X4.number.of.convenience.stores + X5.latitude + X6.longitude
## 
##                                          Df Deviance    AIC
## - X6.longitude                            1   276.11 288.11
## <none>                                        276.05 290.05
## - X1.transaction.date                     1   278.44 290.44
## - X4.number.of.convenience.stores         1   284.80 296.80
## - X5.latitude                             1   297.70 309.70
## - X2.house.age                            1   303.28 315.28
## - X3.distance.to.the.nearest.MRT.station  1   318.97 330.97
## 
## Step:  AIC=288.11
## medHousePriceBin ~ X1.transaction.date + X2.house.age + X3.distance.to.the.nearest.MRT.station + 
##     X4.number.of.convenience.stores + X5.latitude
## 
##                                          Df Deviance    AIC
## <none>                                        276.11 288.11
## - X1.transaction.date                     1   278.46 288.46
## + X6.longitude                            1   276.05 290.05
## - X4.number.of.convenience.stores         1   284.81 294.81
## - X5.latitude                             1   298.19 308.19
## - X2.house.age                            1   303.44 313.44
## - X3.distance.to.the.nearest.MRT.station  1   337.69 347.69
## Start:  AIC=264.43
## medHousePriceBin ~ X1.transaction.date + X2.house.age + X3.distance.to.the.nearest.MRT.station + 
##     X4.number.of.convenience.stores + X5.latitude + X6.longitude
## 
##                                          Df Deviance    AIC
## - X6.longitude                            1   250.75 262.75
## <none>                                        250.43 264.43
## - X1.transaction.date                     1   256.57 268.57
## - X4.number.of.convenience.stores         1   257.29 269.29
## - X5.latitude                             1   276.42 288.42
## - X2.house.age                            1   281.40 293.40
## - X3.distance.to.the.nearest.MRT.station  1   315.76 327.76
## 
## Step:  AIC=262.75
## medHousePriceBin ~ X1.transaction.date + X2.house.age + X3.distance.to.the.nearest.MRT.station + 
##     X4.number.of.convenience.stores + X5.latitude
## 
##                                          Df Deviance    AIC
## <none>                                        250.75 262.75
## + X6.longitude                            1   250.43 264.43
## - X1.transaction.date                     1   256.72 266.72
## - X4.number.of.convenience.stores         1   257.33 267.33
## - X5.latitude                             1   277.19 287.19
## - X2.house.age                            1   281.74 291.74
## - X3.distance.to.the.nearest.MRT.station  1   334.27 344.27
## Start:  AIC=298.95
## medHousePriceBin ~ X1.transaction.date + X2.house.age + X3.distance.to.the.nearest.MRT.station + 
##     X4.number.of.convenience.stores + X5.latitude + X6.longitude
## 
##                                          Df Deviance    AIC
## - X6.longitude                            1   285.27 297.27
## <none>                                        284.95 298.95
## - X1.transaction.date                     1   289.52 301.52
## - X4.number.of.convenience.stores         1   298.06 310.06
## - X5.latitude                             1   303.16 315.16
## - X2.house.age                            1   306.47 318.47
## - X3.distance.to.the.nearest.MRT.station  1   327.37 339.37
## 
## Step:  AIC=297.27
## medHousePriceBin ~ X1.transaction.date + X2.house.age + X3.distance.to.the.nearest.MRT.station + 
##     X4.number.of.convenience.stores + X5.latitude
## 
##                                          Df Deviance    AIC
## <none>                                        285.27 297.27
## + X6.longitude                            1   284.95 298.95
## - X1.transaction.date                     1   289.68 299.68
## - X4.number.of.convenience.stores         1   298.06 308.06
## - X5.latitude                             1   303.90 313.90
## - X2.house.age                            1   307.16 317.16
## - X3.distance.to.the.nearest.MRT.station  1   344.57 354.57
## Start:  AIC=292.54
## medHousePriceBin ~ X1.transaction.date + X2.house.age + X3.distance.to.the.nearest.MRT.station + 
##     X4.number.of.convenience.stores + X5.latitude + X6.longitude
## 
##                                          Df Deviance    AIC
## - X6.longitude                            1   278.67 290.67
## <none>                                        278.54 292.54
## - X1.transaction.date                     1   284.62 296.62
## - X4.number.of.convenience.stores         1   287.68 299.68
## - X5.latitude                             1   298.74 310.74
## - X2.house.age                            1   310.37 322.37
## - X3.distance.to.the.nearest.MRT.station  1   326.06 338.06
## 
## Step:  AIC=290.67
## medHousePriceBin ~ X1.transaction.date + X2.house.age + X3.distance.to.the.nearest.MRT.station + 
##     X4.number.of.convenience.stores + X5.latitude
## 
##                                          Df Deviance    AIC
## <none>                                        278.67 290.67
## + X6.longitude                            1   278.54 292.54
## - X1.transaction.date                     1   284.67 294.67
## - X4.number.of.convenience.stores         1   287.68 297.68
## - X5.latitude                             1   299.42 309.42
## - X2.house.age                            1   310.73 320.73
## - X3.distance.to.the.nearest.MRT.station  1   345.77 355.77
## Start:  AIC=299.9
## medHousePriceBin ~ X1.transaction.date + X2.house.age + X3.distance.to.the.nearest.MRT.station + 
##     X4.number.of.convenience.stores + X5.latitude + X6.longitude
## 
##                                          Df Deviance    AIC
## - X6.longitude                            1   286.07 298.07
## <none>                                        285.90 299.90
## - X1.transaction.date                     1   288.64 300.64
## - X4.number.of.convenience.stores         1   295.19 307.19
## - X5.latitude                             1   309.99 321.99
## - X2.house.age                            1   313.17 325.17
## - X3.distance.to.the.nearest.MRT.station  1   329.96 341.96
## 
## Step:  AIC=298.07
## medHousePriceBin ~ X1.transaction.date + X2.house.age + X3.distance.to.the.nearest.MRT.station + 
##     X4.number.of.convenience.stores + X5.latitude
## 
##                                          Df Deviance    AIC
## <none>                                        286.07 298.07
## - X1.transaction.date                     1   288.75 298.75
## + X6.longitude                            1   285.90 299.90
## - X4.number.of.convenience.stores         1   295.19 305.19
## - X5.latitude                             1   310.75 320.75
## - X2.house.age                            1   313.76 323.76
## - X3.distance.to.the.nearest.MRT.station  1   347.30 357.30
## Start:  AIC=299.55
## medHousePriceBin ~ X1.transaction.date + X2.house.age + X3.distance.to.the.nearest.MRT.station + 
##     X4.number.of.convenience.stores + X5.latitude + X6.longitude
## 
##                                          Df Deviance    AIC
## - X6.longitude                            1   285.94 297.94
## <none>                                        285.55 299.55
## - X1.transaction.date                     1   290.16 302.16
## - X4.number.of.convenience.stores         1   296.08 308.08
## - X5.latitude                             1   305.64 317.64
## - X2.house.age                            1   310.43 322.43
## - X3.distance.to.the.nearest.MRT.station  1   329.63 341.63
## 
## Step:  AIC=297.94
## medHousePriceBin ~ X1.transaction.date + X2.house.age + X3.distance.to.the.nearest.MRT.station + 
##     X4.number.of.convenience.stores + X5.latitude
## 
##                                          Df Deviance    AIC
## <none>                                        285.94 297.94
## + X6.longitude                            1   285.55 299.55
## - X1.transaction.date                     1   290.37 300.37
## - X4.number.of.convenience.stores         1   296.14 306.14
## - X5.latitude                             1   306.49 316.49
## - X2.house.age                            1   311.14 321.14
## - X3.distance.to.the.nearest.MRT.station  1   345.91 355.91
## Start:  AIC=301.03
## medHousePriceBin ~ X1.transaction.date + X2.house.age + X3.distance.to.the.nearest.MRT.station + 
##     X4.number.of.convenience.stores + X5.latitude + X6.longitude
## 
##                                          Df Deviance    AIC
## - X6.longitude                            1   287.16 299.16
## <none>                                        287.03 301.03
## - X1.transaction.date                     1   290.23 302.23
## - X4.number.of.convenience.stores         1   296.86 308.86
## - X5.latitude                             1   305.59 317.59
## - X2.house.age                            1   310.06 322.06
## - X3.distance.to.the.nearest.MRT.station  1   334.42 346.42
## 
## Step:  AIC=299.16
## medHousePriceBin ~ X1.transaction.date + X2.house.age + X3.distance.to.the.nearest.MRT.station + 
##     X4.number.of.convenience.stores + X5.latitude
## 
##                                          Df Deviance    AIC
## <none>                                        287.16 299.16
## - X1.transaction.date                     1   290.30 300.30
## + X6.longitude                            1   287.03 301.03
## - X4.number.of.convenience.stores         1   296.88 306.88
## - X5.latitude                             1   306.13 316.13
## - X2.house.age                            1   310.45 320.45
## - X3.distance.to.the.nearest.MRT.station  1   352.85 362.85
## Start:  AIC=301.42
## medHousePriceBin ~ X1.transaction.date + X2.house.age + X3.distance.to.the.nearest.MRT.station + 
##     X4.number.of.convenience.stores + X5.latitude + X6.longitude
## 
##                                          Df Deviance    AIC
## - X6.longitude                            1   287.75 299.75
## <none>                                        287.42 301.42
## - X1.transaction.date                     1   291.41 303.41
## - X4.number.of.convenience.stores         1   297.69 309.69
## - X5.latitude                             1   306.90 318.90
## - X2.house.age                            1   311.93 323.93
## - X3.distance.to.the.nearest.MRT.station  1   333.57 345.57
## 
## Step:  AIC=299.75
## medHousePriceBin ~ X1.transaction.date + X2.house.age + X3.distance.to.the.nearest.MRT.station + 
##     X4.number.of.convenience.stores + X5.latitude
## 
##                                          Df Deviance    AIC
## <none>                                        287.75 299.75
## + X6.longitude                            1   287.42 301.42
## - X1.transaction.date                     1   291.64 301.64
## - X4.number.of.convenience.stores         1   297.69 307.69
## - X5.latitude                             1   307.83 317.83
## - X2.house.age                            1   312.58 322.58
## - X3.distance.to.the.nearest.MRT.station  1   349.05 359.05
## Start:  AIC=302.77
## medHousePriceBin ~ X1.transaction.date + X2.house.age + X3.distance.to.the.nearest.MRT.station + 
##     X4.number.of.convenience.stores + X5.latitude + X6.longitude
## 
##                                          Df Deviance    AIC
## - X6.longitude                            1   288.80 300.80
## <none>                                        288.77 302.77
## - X1.transaction.date                     1   293.70 305.70
## - X4.number.of.convenience.stores         1   297.50 309.50
## - X5.latitude                             1   312.84 324.84
## - X2.house.age                            1   322.49 334.49
## - X3.distance.to.the.nearest.MRT.station  1   332.82 344.82
## 
## Step:  AIC=300.8
## medHousePriceBin ~ X1.transaction.date + X2.house.age + X3.distance.to.the.nearest.MRT.station + 
##     X4.number.of.convenience.stores + X5.latitude
## 
##                                          Df Deviance    AIC
## <none>                                        288.80 300.80
## + X6.longitude                            1   288.77 302.77
## - X1.transaction.date                     1   293.70 303.70
## - X4.number.of.convenience.stores         1   297.64 307.64
## - X5.latitude                             1   313.38 323.38
## - X2.house.age                            1   322.81 332.81
## - X3.distance.to.the.nearest.MRT.station  1   350.97 360.97
## Start:  AIC=291.73
## medHousePriceBin ~ X1.transaction.date + X2.house.age + X3.distance.to.the.nearest.MRT.station + 
##     X4.number.of.convenience.stores + X5.latitude + X6.longitude
## 
##                                          Df Deviance    AIC
## - X6.longitude                            1   278.10 290.10
## <none>                                        277.73 291.73
## - X1.transaction.date                     1   280.63 292.63
## - X4.number.of.convenience.stores         1   290.68 302.68
## - X5.latitude                             1   300.24 312.24
## - X2.house.age                            1   308.75 320.75
## - X3.distance.to.the.nearest.MRT.station  1   320.21 332.21
## 
## Step:  AIC=290.1
## medHousePriceBin ~ X1.transaction.date + X2.house.age + X3.distance.to.the.nearest.MRT.station + 
##     X4.number.of.convenience.stores + X5.latitude
## 
##                                          Df Deviance    AIC
## <none>                                        278.10 290.10
## - X1.transaction.date                     1   280.89 290.89
## + X6.longitude                            1   277.73 291.73
## - X4.number.of.convenience.stores         1   290.68 300.68
## - X5.latitude                             1   301.05 311.05
## - X2.house.age                            1   309.45 319.45
## - X3.distance.to.the.nearest.MRT.station  1   335.06 345.06
## Start:  AIC=302.77
## medHousePriceBin ~ X1.transaction.date + X2.house.age + X3.distance.to.the.nearest.MRT.station + 
##     X4.number.of.convenience.stores + X5.latitude + X6.longitude
## 
##                                          Df Deviance    AIC
## - X6.longitude                            1   288.84 300.84
## <none>                                        288.77 302.77
## - X1.transaction.date                     1   291.89 303.89
## - X4.number.of.convenience.stores         1   299.13 311.13
## - X5.latitude                             1   313.00 325.00
## - X2.house.age                            1   316.93 328.93
## - X3.distance.to.the.nearest.MRT.station  1   332.48 344.48
## 
## Step:  AIC=300.84
## medHousePriceBin ~ X1.transaction.date + X2.house.age + X3.distance.to.the.nearest.MRT.station + 
##     X4.number.of.convenience.stores + X5.latitude
## 
##                                          Df Deviance    AIC
## <none>                                        288.84 300.84
## - X1.transaction.date                     1   291.91 301.91
## + X6.longitude                            1   288.77 302.77
## - X4.number.of.convenience.stores         1   299.17 309.17
## - X5.latitude                             1   313.52 323.52
## - X2.house.age                            1   317.23 327.23
## - X3.distance.to.the.nearest.MRT.station  1   350.16 360.16
## Start:  AIC=299.24
## medHousePriceBin ~ X1.transaction.date + X2.house.age + X3.distance.to.the.nearest.MRT.station + 
##     X4.number.of.convenience.stores + X5.latitude + X6.longitude
## 
##                                          Df Deviance    AIC
## - X6.longitude                            1   285.74 297.74
## <none>                                        285.24 299.24
## - X1.transaction.date                     1   288.20 300.20
## - X4.number.of.convenience.stores         1   290.33 302.33
## - X5.latitude                             1   305.97 317.97
## - X2.house.age                            1   312.34 324.34
## - X3.distance.to.the.nearest.MRT.station  1   339.14 351.14
## 
## Step:  AIC=297.74
## medHousePriceBin ~ X1.transaction.date + X2.house.age + X3.distance.to.the.nearest.MRT.station + 
##     X4.number.of.convenience.stores + X5.latitude
## 
##                                          Df Deviance    AIC
## <none>                                        285.74 297.74
## - X1.transaction.date                     1   288.60 298.60
## + X6.longitude                            1   285.24 299.24
## - X4.number.of.convenience.stores         1   290.50 300.50
## - X5.latitude                             1   307.29 317.29
## - X2.house.age                            1   313.13 323.13
## - X3.distance.to.the.nearest.MRT.station  1   357.05 367.05
## Start:  AIC=300.93
## medHousePriceBin ~ X1.transaction.date + X2.house.age + X3.distance.to.the.nearest.MRT.station + 
##     X4.number.of.convenience.stores + X5.latitude + X6.longitude
## 
##                                          Df Deviance    AIC
## - X6.longitude                            1   287.28 299.28
## <none>                                        286.93 300.93
## - X1.transaction.date                     1   291.18 303.18
## - X4.number.of.convenience.stores         1   293.19 305.19
## - X5.latitude                             1   303.52 315.52
## - X2.house.age                            1   312.54 324.54
## - X3.distance.to.the.nearest.MRT.station  1   342.55 354.55
## 
## Step:  AIC=299.28
## medHousePriceBin ~ X1.transaction.date + X2.house.age + X3.distance.to.the.nearest.MRT.station + 
##     X4.number.of.convenience.stores + X5.latitude
## 
##                                          Df Deviance    AIC
## <none>                                        287.28 299.28
## + X6.longitude                            1   286.93 300.93
## - X1.transaction.date                     1   291.40 301.40
## - X4.number.of.convenience.stores         1   293.20 303.20
## - X5.latitude                             1   304.24 314.24
## - X2.house.age                            1   313.05 323.05
## - X3.distance.to.the.nearest.MRT.station  1   361.60 371.60
## Start:  AIC=299.58
## medHousePriceBin ~ X1.transaction.date + X2.house.age + X3.distance.to.the.nearest.MRT.station + 
##     X4.number.of.convenience.stores + X5.latitude + X6.longitude
## 
##                                          Df Deviance    AIC
## - X6.longitude                            1   285.59 297.59
## <none>                                        285.58 299.58
## - X1.transaction.date                     1   291.40 303.40
## - X4.number.of.convenience.stores         1   294.74 306.74
## - X5.latitude                             1   309.99 321.99
## - X2.house.age                            1   315.86 327.86
## - X3.distance.to.the.nearest.MRT.station  1   326.53 338.53
## 
## Step:  AIC=297.59
## medHousePriceBin ~ X1.transaction.date + X2.house.age + X3.distance.to.the.nearest.MRT.station + 
##     X4.number.of.convenience.stores + X5.latitude
## 
##                                          Df Deviance    AIC
## <none>                                        285.59 297.59
## + X6.longitude                            1   285.58 299.58
## - X1.transaction.date                     1   291.41 301.41
## - X4.number.of.convenience.stores         1   294.90 304.90
## - X5.latitude                             1   310.27 320.27
## - X2.house.age                            1   316.04 326.04
## - X3.distance.to.the.nearest.MRT.station  1   343.95 353.95
## Start:  AIC=294.49
## medHousePriceBin ~ X1.transaction.date + X2.house.age + X3.distance.to.the.nearest.MRT.station + 
##     X4.number.of.convenience.stores + X5.latitude + X6.longitude
## 
##                                          Df Deviance    AIC
## - X6.longitude                            1   280.68 292.68
## <none>                                        280.49 294.49
## - X1.transaction.date                     1   287.31 299.31
## - X4.number.of.convenience.stores         1   291.83 303.83
## - X5.latitude                             1   303.12 315.12
## - X2.house.age                            1   307.53 319.53
## - X3.distance.to.the.nearest.MRT.station  1   327.56 339.56
## 
## Step:  AIC=292.68
## medHousePriceBin ~ X1.transaction.date + X2.house.age + X3.distance.to.the.nearest.MRT.station + 
##     X4.number.of.convenience.stores + X5.latitude
## 
##                                          Df Deviance    AIC
## <none>                                        280.68 292.68
## + X6.longitude                            1   280.49 294.49
## - X1.transaction.date                     1   287.39 297.39
## - X4.number.of.convenience.stores         1   291.84 301.84
## - X5.latitude                             1   303.87 313.87
## - X2.house.age                            1   308.05 318.05
## - X3.distance.to.the.nearest.MRT.station  1   344.75 354.75
\end{verbatim}

\begin{Shaded}
\begin{Highlighting}[]
\KeywordTok{mean}\NormalTok{(recallk.step)}
\end{Highlighting}
\end{Shaded}

\begin{verbatim}
## [1] 87.42318
\end{verbatim}

\begin{Shaded}
\begin{Highlighting}[]
\KeywordTok{mean}\NormalTok{(error_ratek.step)}
\end{Highlighting}
\end{Shaded}

\begin{verbatim}
## [1] 17.60141
\end{verbatim}

\begin{Shaded}
\begin{Highlighting}[]
\KeywordTok{mean}\NormalTok{(accuracyk.step)}
\end{Highlighting}
\end{Shaded}

\begin{verbatim}
## [1] 82.39859
\end{verbatim}

\begin{Shaded}
\begin{Highlighting}[]
\KeywordTok{mean}\NormalTok{(precisionk.step)}
\end{Highlighting}
\end{Shaded}

\begin{verbatim}
## [1] 80.56085
\end{verbatim}

En utilisant la validation croisée, on remarque que nous avons une
performance globale qui se situe autour de 80\%, avec un taux d'erreur
autour de 20\%.

\hypertarget{i.3-ridge-regression}{%
\subsection{I.3 Ridge regression}\label{i.3-ridge-regression}}

On effectue ici une regression ridge. \#\#\# a )Ridge model

\begin{Shaded}
\begin{Highlighting}[]
\NormalTok{grid =}\StringTok{ }\DecValTok{10}\OperatorTok{^}\KeywordTok{seq}\NormalTok{(}\DecValTok{5}\NormalTok{,}\OperatorTok{-}\DecValTok{2}\NormalTok{,}\DataTypeTok{length =} \DecValTok{100}\NormalTok{) }\CommentTok{# sequence des lambda}
\NormalTok{model.ridge <-}\StringTok{ }\KeywordTok{glmnet}\NormalTok{(X.train,Y.train,}\DataTypeTok{alpha=}\DecValTok{0}\NormalTok{,}\DataTypeTok{lambda =}\NormalTok{ grid,}\DataTypeTok{family =} \StringTok{"binomial"}\NormalTok{) }\CommentTok{# model}
\KeywordTok{plot}\NormalTok{(model.ridge,}\DataTypeTok{xvar=}\StringTok{"lambda"}\NormalTok{,}\DataTypeTok{type=}\StringTok{"l"}\NormalTok{,}\DataTypeTok{col=}\DecValTok{1}\OperatorTok{:}\KeywordTok{nrow}\NormalTok{(tab.train)}\OperatorTok{-}\DecValTok{1}\NormalTok{);}\KeywordTok{legend}\NormalTok{(}\StringTok{"topright"}\NormalTok{                                                            ,}\DataTypeTok{legend=}\KeywordTok{colnames}\NormalTok{(tab.train[,}\DecValTok{1}\OperatorTok{:}\KeywordTok{ncol}\NormalTok{(tab.train)}\OperatorTok{-}\DecValTok{1}\NormalTok{]), }\DataTypeTok{col=}\DecValTok{1}\OperatorTok{:}\DecValTok{10}\NormalTok{, }\DataTypeTok{lty=}\DecValTok{1}\NormalTok{)}
\end{Highlighting}
\end{Shaded}

\includegraphics{TP3_files/figure-latex/unnamed-chunk-18-1.pdf}

\hypertarget{b-selection-du-lambda-par-cross-validation}{%
\subsubsection{\texorpdfstring{b) Selection du \(\lambda\) par cross
validation}{b) Selection du \textbackslash lambda par cross validation}}\label{b-selection-du-lambda-par-cross-validation}}

\begin{Shaded}
\begin{Highlighting}[]
\CommentTok{####################### cross validation}
\NormalTok{ridge.cv.out<-}\KeywordTok{cv.glmnet}\NormalTok{(X.train, Y.train, }\DataTypeTok{alpha =} \DecValTok{0}\NormalTok{,}\DataTypeTok{nfolds =} \DecValTok{10}\NormalTok{,}\DataTypeTok{family =} \StringTok{"binomial"}\NormalTok{); ridge.cv.out }\CommentTok{# on sélectionne la meilleure valeur de lambda par validation croisée}
\end{Highlighting}
\end{Shaded}

\begin{verbatim}
## 
## Call:  cv.glmnet(x = X.train, y = Y.train, nfolds = 10, alpha = 0, family = "binomial") 
## 
## Measure: Binomial Deviance 
## 
##      Lambda Measure      SE Nonzero
## min 0.02884  0.7427 0.03697       6
## 1se 0.05531  0.7756 0.03343       6
\end{verbatim}

\begin{Shaded}
\begin{Highlighting}[]
\NormalTok{ridge.lamb.min<-ridge.cv.out}\OperatorTok{$}\NormalTok{lambda.min }\CommentTok{# le meilleur lambda est celui qui produit the min MSE}
\end{Highlighting}
\end{Shaded}

On selectionne le modele le lambda qui minimise le MSE pour notre
modèle. On effectue 10 folds.

\hypertarget{c-prediction-du-model-ridge}{%
\subsubsection{c) Prediction du model
Ridge}\label{c-prediction-du-model-ridge}}

\begin{Shaded}
\begin{Highlighting}[]
\NormalTok{ridge.pred <-}\StringTok{ }\KeywordTok{predict}\NormalTok{(model.ridge, }\DataTypeTok{s =}\NormalTok{ ridge.lamb.min, }\DataTypeTok{newx =}\NormalTok{ X.test,}\DataTypeTok{type =} \StringTok{'response'}\NormalTok{)}
\NormalTok{Y.pred.ridge =}\StringTok{  }\KeywordTok{as.integer}\NormalTok{(ridge.pred }\OperatorTok{>=}\StringTok{ }\NormalTok{Threshold)}
\NormalTok{confusion_matrix.ridge =}\StringTok{ }\KeywordTok{table}\NormalTok{(Y.pred.ridge,Y.test) }\CommentTok{# matrice de confusion}
\NormalTok{confusion_matrix.ridge}
\end{Highlighting}
\end{Shaded}

\begin{verbatim}
##             Y.test
## Y.pred.ridge  0  1
##            0 31  8
##            1 10 28
\end{verbatim}

\hypertarget{d-performance-du-model}{%
\subsubsection{d) Performance du model}\label{d-performance-du-model}}

\begin{Shaded}
\begin{Highlighting}[]
\NormalTok{pred.accuracy.ridge =}\StringTok{ }\KeywordTok{sum}\NormalTok{(}\KeywordTok{diag}\NormalTok{(confusion_matrix.ridge))}\OperatorTok{/}\KeywordTok{sum}\NormalTok{(confusion_matrix.ridge)}\OperatorTok{*}\DecValTok{100}\CommentTok{#   prediction accuracy}
\NormalTok{pred.recall.ridge =}\StringTok{ }\NormalTok{confusion_matrix.ridge[}\DecValTok{2}\NormalTok{,}\DecValTok{2}\NormalTok{]}\OperatorTok{/}\KeywordTok{sum}\NormalTok{(confusion_matrix.ridge[,}\DecValTok{2}\NormalTok{])}\OperatorTok{*}\DecValTok{100} \CommentTok{# proportion des hauts prix bien predit }
\NormalTok{pred.specifity.ridge =}\StringTok{ }\NormalTok{confusion_matrix.ridge[}\DecValTok{1}\NormalTok{,}\DecValTok{1}\NormalTok{]}\OperatorTok{/}\KeywordTok{sum}\NormalTok{(confusion_matrix.ridge[,}\DecValTok{1}\NormalTok{])}\OperatorTok{*}\DecValTok{100} \CommentTok{# proportion des bas prix bien predits}
\NormalTok{pred.precision.ridge =}\StringTok{ }\NormalTok{confusion_matrix.ridge[}\DecValTok{2}\NormalTok{,}\DecValTok{2}\NormalTok{]}\OperatorTok{/}\KeywordTok{sum}\NormalTok{(confusion_matrix.ridge[}\DecValTok{2}\NormalTok{,])}\OperatorTok{*}\DecValTok{100}
\NormalTok{pred.error_rate.ridge =}
\KeywordTok{sum}\NormalTok{(}\KeywordTok{diag}\NormalTok{(confusion_matrix.ridge[}\DecValTok{1}\OperatorTok{:}\DecValTok{2}\NormalTok{,}\DecValTok{2}\OperatorTok{:}\DecValTok{1}\NormalTok{]))}\OperatorTok{/}\KeywordTok{sum}\NormalTok{(confusion_matrix.ridge) }\OperatorTok{*}\DecValTok{100} \CommentTok{# proportion des mauvaises predictions}

\NormalTok{pred.accuracy.ridge}
\end{Highlighting}
\end{Shaded}

\begin{verbatim}
## [1] 76.62338
\end{verbatim}

\begin{Shaded}
\begin{Highlighting}[]
\NormalTok{pred.recall.ridge}
\end{Highlighting}
\end{Shaded}

\begin{verbatim}
## [1] 77.77778
\end{verbatim}

\begin{Shaded}
\begin{Highlighting}[]
\NormalTok{pred.specifity.ridge}
\end{Highlighting}
\end{Shaded}

\begin{verbatim}
## [1] 75.60976
\end{verbatim}

\begin{Shaded}
\begin{Highlighting}[]
\NormalTok{pred.error_rate.ridge}
\end{Highlighting}
\end{Shaded}

\begin{verbatim}
## [1] 23.37662
\end{verbatim}

On retrouve une performance de 76\% sur le modèle. Le modèle predit bien
la classe des hauts prix à 77\% et celle de pas prix à 75\%.

\hypertarget{e-k-folds-le-model-ridge}{%
\subsubsection{e) k-folds le model
ridge}\label{e-k-folds-le-model-ridge}}

\begin{Shaded}
\begin{Highlighting}[]
\CommentTok{##shuffling}
\KeywordTok{set.seed}\NormalTok{(}\DecValTok{1234}\NormalTok{)}
\NormalTok{rows <-}\StringTok{ }\KeywordTok{sample}\NormalTok{(}\KeywordTok{nrow}\NormalTok{(tabmed)) }\CommentTok{# melange des rows du dataset}
\NormalTok{tabmedk <-}\StringTok{ }\NormalTok{tabmed[rows, ]}
\CommentTok{## folds}
\NormalTok{k =}\StringTok{ }\DecValTok{15} \CommentTok{#as.integer(1/(1-r)) ## fold number}
\NormalTok{fold =}\StringTok{ }\KeywordTok{cut}\NormalTok{(}\KeywordTok{seq}\NormalTok{(}\DecValTok{1}\NormalTok{,}\KeywordTok{nrow}\NormalTok{(tabmed)), }\DataTypeTok{breaks =}\NormalTok{ k,}\DataTypeTok{labels =} \OtherTok{FALSE}\NormalTok{) }\CommentTok{#organisations des lignes du}
\CommentTok{#dataset en plusieurs fold}
\CommentTok{#initialisation des vectueurs qui va contenier les performances a chaque etape}
\NormalTok{accuracyk.ridge =}\StringTok{ }\KeywordTok{c}\NormalTok{()}
\NormalTok{recallk.ridge =}\StringTok{ }\KeywordTok{c}\NormalTok{()}
\NormalTok{precisionk.ridge =}\StringTok{ }\KeywordTok{c}\NormalTok{()}
\NormalTok{error_ratek.ridge =}\StringTok{ }\KeywordTok{c}\NormalTok{()}
\ControlFlowTok{for}\NormalTok{ (i }\ControlFlowTok{in} \DecValTok{1}\OperatorTok{:}\NormalTok{k) \{}
\NormalTok{  test_rows =}\StringTok{ }\KeywordTok{which}\NormalTok{(fold }\OperatorTok{==}\StringTok{ }\NormalTok{i,}\DataTypeTok{arr.ind =} \OtherTok{TRUE}\NormalTok{) }
\NormalTok{  tab.testk =}\StringTok{ }\NormalTok{tabmedk[test_rows,]}
\NormalTok{  tab.traink =}\StringTok{ }\NormalTok{tabmedk[}\OperatorTok{-}\NormalTok{test_rows,]}
\NormalTok{  X.traink =}\StringTok{ }\KeywordTok{as.matrix}\NormalTok{(tab.traink[,}\OperatorTok{-}\KeywordTok{dim}\NormalTok{(tab.traink)[}\DecValTok{2}\NormalTok{]])}
\NormalTok{  X.testk =}\StringTok{ }\KeywordTok{as.matrix}\NormalTok{(tab.testk[,}\OperatorTok{-}\KeywordTok{dim}\NormalTok{(tab.testk)[}\DecValTok{2}\NormalTok{]])}
\NormalTok{  Y.testk =}\StringTok{ }\NormalTok{tab.testk}\OperatorTok{$}\NormalTok{medHousePriceBin}
\NormalTok{  Y.traink =}\StringTok{ }\NormalTok{tab.traink}\OperatorTok{$}\NormalTok{medHousePriceBin}
  \CommentTok{### regression logistic }
\NormalTok{  model.ridgek <-}\StringTok{ }\KeywordTok{glmnet}\NormalTok{(X.traink,Y.traink,}\DataTypeTok{alpha=}\DecValTok{0}\NormalTok{,}\DataTypeTok{lambda =}\NormalTok{ grid,}\DataTypeTok{family =} \StringTok{"binomial"}\NormalTok{)}
  \CommentTok{### prediction}
  \CommentTok{####################### cross validation}
\NormalTok{  ridge.cv.out<-}\KeywordTok{cv.glmnet}\NormalTok{(X.traink, Y.traink, }\DataTypeTok{alpha =} \DecValTok{0}\NormalTok{,}\DataTypeTok{nfolds =} \DecValTok{10}\NormalTok{,}\DataTypeTok{family =} \StringTok{"binomial"}\NormalTok{) }\CommentTok{# on sélectionne la meilleure valeur de lambda par validation croisée}
\NormalTok{  ridge.lamb.mink<-ridge.cv.out}\OperatorTok{$}\NormalTok{lambda.min }\CommentTok{# le meilleur lambda est celui qui produit the min MSE}
  
\NormalTok{  prob <-}\StringTok{ }\KeywordTok{predict}\NormalTok{(model.ridgek, }\DataTypeTok{s =}\NormalTok{ ridge.lamb.mink, }\DataTypeTok{newx =}\NormalTok{ X.testk,}\DataTypeTok{type =} \StringTok{'response'}\NormalTok{)}
\NormalTok{  Y.pred.ridge =}\StringTok{  }\KeywordTok{as.integer}\NormalTok{(prob }\OperatorTok{>=}\StringTok{ }\NormalTok{Threshold) }
\NormalTok{  confusion_matrix =}\StringTok{ }\KeywordTok{table}\NormalTok{(Y.pred.ridge,tab.testk}\OperatorTok{$}\NormalTok{medHousePriceBin)}
\NormalTok{  accuracyk.ridge[i] =}\StringTok{ }\KeywordTok{sum}\NormalTok{(}\KeywordTok{diag}\NormalTok{(confusion_matrix))}\OperatorTok{/}\KeywordTok{sum}\NormalTok{(confusion_matrix)}\OperatorTok{*}\DecValTok{100}
\NormalTok{  recallk.ridge[i] =}\StringTok{ }\NormalTok{confusion_matrix[}\DecValTok{2}\NormalTok{,}\DecValTok{2}\NormalTok{]}\OperatorTok{/}\KeywordTok{sum}\NormalTok{(confusion_matrix[,}\DecValTok{2}\NormalTok{])}\OperatorTok{*}\DecValTok{100}
\NormalTok{  precisionk.ridge[i] =confusion_matrix[}\DecValTok{2}\NormalTok{,}\DecValTok{2}\NormalTok{]}\OperatorTok{/}\KeywordTok{sum}\NormalTok{(confusion_matrix[}\DecValTok{2}\NormalTok{,])}\OperatorTok{*}\DecValTok{100}
\NormalTok{  error_ratek.ridge[i] =}\StringTok{ }\KeywordTok{sum}\NormalTok{(}\KeywordTok{diag}\NormalTok{(confusion_matrix[}\DecValTok{1}\OperatorTok{:}\DecValTok{2}\NormalTok{,}\DecValTok{2}\OperatorTok{:}\DecValTok{1}\NormalTok{]))}\OperatorTok{/}\KeywordTok{sum}\NormalTok{(confusion_matrix) }\OperatorTok{*}\DecValTok{100} 
\NormalTok{\}}

\KeywordTok{mean}\NormalTok{(recallk.ridge)}
\end{Highlighting}
\end{Shaded}

\begin{verbatim}
## [1] 86.68707
\end{verbatim}

\begin{Shaded}
\begin{Highlighting}[]
\KeywordTok{mean}\NormalTok{(error_ratek.ridge)}
\end{Highlighting}
\end{Shaded}

\begin{verbatim}
## [1] 18.61552
\end{verbatim}

\begin{Shaded}
\begin{Highlighting}[]
\KeywordTok{mean}\NormalTok{(accuracyk.ridge)}
\end{Highlighting}
\end{Shaded}

\begin{verbatim}
## [1] 81.38448
\end{verbatim}

\begin{Shaded}
\begin{Highlighting}[]
\KeywordTok{mean}\NormalTok{(precisionk.ridge)}
\end{Highlighting}
\end{Shaded}

\begin{verbatim}
## [1] 78.41565
\end{verbatim}

\hypertarget{i.4-moduxe8le-lasso}{%
\subsection{I.4) Modèle Lasso}\label{i.4-moduxe8le-lasso}}

\hypertarget{a-model-lasso}{%
\subsubsection{a) Model Lasso}\label{a-model-lasso}}

\begin{Shaded}
\begin{Highlighting}[]
\NormalTok{grid =}\StringTok{ }\DecValTok{10}\OperatorTok{^}\KeywordTok{seq}\NormalTok{(}\DecValTok{5}\NormalTok{,}\OperatorTok{-}\DecValTok{2}\NormalTok{,}\DataTypeTok{length =} \DecValTok{100}\NormalTok{) }\CommentTok{# sequence des lambda}
\NormalTok{lasso=}\KeywordTok{glmnet}\NormalTok{(X.train,Y.train,}\DataTypeTok{alpha=}\DecValTok{1}\NormalTok{,}\DataTypeTok{lambda=}\NormalTok{grid,}\DataTypeTok{family =} \StringTok{"binomial"}\NormalTok{, }\DataTypeTok{standardize=}\OtherTok{FALSE}\NormalTok{)}
\KeywordTok{plot}\NormalTok{(lasso,}\DataTypeTok{xvar=}\StringTok{"lambda"}\NormalTok{,}\DataTypeTok{type=}\StringTok{"l"}\NormalTok{,}\DataTypeTok{col=}\DecValTok{1}\OperatorTok{:}\DecValTok{10}\NormalTok{);}\KeywordTok{legend}\NormalTok{(}\StringTok{"topright"}\NormalTok{,}\DataTypeTok{legend=}\KeywordTok{colnames}\NormalTok{(X.train[,}\DecValTok{1}\OperatorTok{:}\KeywordTok{ncol}\NormalTok{(X.train)]), }\DataTypeTok{col=}\DecValTok{1}\OperatorTok{:}\DecValTok{10}\NormalTok{, }\DataTypeTok{lty=}\DecValTok{1}\NormalTok{)}
\end{Highlighting}
\end{Shaded}

\includegraphics{TP3_files/figure-latex/unnamed-chunk-23-1.pdf} On voit
que plus lambda est grand, plus le nombre de variables sélectionnées est
petit.

\hypertarget{b-selection-du-lambda-par-cross-validation-1}{%
\subsubsection{\texorpdfstring{b) Selection du \(\lambda\) par cross
validation}{b) Selection du \textbackslash lambda par cross validation}}\label{b-selection-du-lambda-par-cross-validation-1}}

\begin{Shaded}
\begin{Highlighting}[]
\NormalTok{lasso.cv.out=}\KeywordTok{cv.glmnet}\NormalTok{(X.train, Y.train, }\DataTypeTok{alpha =} \DecValTok{1}\NormalTok{,}\DataTypeTok{nfolds =} \DecValTok{10}\NormalTok{,}\DataTypeTok{family =} \StringTok{"binomial"}\NormalTok{); lasso.cv.out}
\end{Highlighting}
\end{Shaded}

\begin{verbatim}
## 
## Call:  cv.glmnet(x = X.train, y = Y.train, nfolds = 10, alpha = 1, family = "binomial") 
## 
## Measure: Binomial Deviance 
## 
##      Lambda Measure      SE Nonzero
## min 0.00923  0.7484 0.09288       5
## 1se 0.04923  0.8361 0.05650       4
\end{verbatim}

\begin{Shaded}
\begin{Highlighting}[]
\NormalTok{lasso.min=lasso.cv.out}\OperatorTok{$}\NormalTok{lambda.min }\CommentTok{# on sélectionne le plus petit lambda qui minimise le MSE }
\end{Highlighting}
\end{Shaded}

On voit que plus lambda est grand, plus le nombre de variables
sélectionnées est petit.

\begin{Shaded}
\begin{Highlighting}[]
\NormalTok{lasso.cv.out=}\KeywordTok{cv.glmnet}\NormalTok{(X.train, Y.train, }\DataTypeTok{alpha =} \DecValTok{1}\NormalTok{,}\DataTypeTok{nfolds =} \DecValTok{10}\NormalTok{,}\DataTypeTok{family =} \StringTok{"binomial"}\NormalTok{); lasso.cv.out}
\end{Highlighting}
\end{Shaded}

\begin{verbatim}
## 
## Call:  cv.glmnet(x = X.train, y = Y.train, nfolds = 10, alpha = 1, family = "binomial") 
## 
## Measure: Binomial Deviance 
## 
##      Lambda Measure      SE Nonzero
## min 0.00636  0.7230 0.08094       5
## 1se 0.04088  0.8038 0.06328       4
\end{verbatim}

\begin{Shaded}
\begin{Highlighting}[]
\NormalTok{lasso.min=lasso.cv.out}\OperatorTok{$}\NormalTok{lambda.min }\CommentTok{# on sélectionne le plus petit lambda qui minimise le MSE }
\end{Highlighting}
\end{Shaded}

On effectue une prédiction du modèle lasso avec le plus petit lambda
minimisant le MSE

\hypertarget{c-prediction-du-model-lasso}{%
\subsubsection{c) Prediction du model
Lasso}\label{c-prediction-du-model-lasso}}

\begin{Shaded}
\begin{Highlighting}[]
\NormalTok{prob=}\StringTok{ }\KeywordTok{predict}\NormalTok{(lasso, }\DataTypeTok{s=}\NormalTok{lasso.min, }\DataTypeTok{newx=}\NormalTok{X.test, }\DataTypeTok{type=}\StringTok{"response"}\NormalTok{)}
\NormalTok{lasso.pred=}\KeywordTok{as.numeric}\NormalTok{(prob}\OperatorTok{>=}\NormalTok{Threshold) }\CommentTok{# transforme les données en données binaire avec un seuil de 1/2}
\end{Highlighting}
\end{Shaded}

\hypertarget{d-performance-du-model-1}{%
\subsubsection{d) Performance du model}\label{d-performance-du-model-1}}

\begin{Shaded}
\begin{Highlighting}[]
\NormalTok{confusion_matrix.lasso =}\StringTok{ }\KeywordTok{table}\NormalTok{(lasso.pred,Y.test) }\CommentTok{# on crée la matrice de confusion }
\NormalTok{confusion_matrix.lasso}
\end{Highlighting}
\end{Shaded}

\begin{verbatim}
##           Y.test
## lasso.pred  0  1
##          0 30  7
##          1 11 29
\end{verbatim}

\begin{Shaded}
\begin{Highlighting}[]
\NormalTok{pred.accuracy.lasso =}\StringTok{ }\KeywordTok{sum}\NormalTok{(}\KeywordTok{diag}\NormalTok{(confusion_matrix.lasso))}\OperatorTok{/}\KeywordTok{sum}\NormalTok{(confusion_matrix.lasso)}\OperatorTok{*}\DecValTok{100}\CommentTok{#   prediction accuracy}
\NormalTok{pred.recall.lasso =}\StringTok{ }\NormalTok{confusion_matrix.lasso[}\DecValTok{2}\NormalTok{,}\DecValTok{2}\NormalTok{]}\OperatorTok{/}\KeywordTok{sum}\NormalTok{(confusion_matrix.lasso[,}\DecValTok{2}\NormalTok{])}\OperatorTok{*}\DecValTok{100} \CommentTok{# proportion des hauts prix bien predit }
\NormalTok{pred.specifity.lasso =}\StringTok{ }\NormalTok{confusion_matrix.lasso[}\DecValTok{1}\NormalTok{,}\DecValTok{1}\NormalTok{]}\OperatorTok{/}\KeywordTok{sum}\NormalTok{(confusion_matrix.lasso[,}\DecValTok{1}\NormalTok{])}\OperatorTok{*}\DecValTok{100} \CommentTok{# proportion des bas prix bien predits}
\NormalTok{pred.precision.lasso =}\StringTok{ }\NormalTok{confusion_matrix.lasso[}\DecValTok{2}\NormalTok{,}\DecValTok{2}\NormalTok{]}\OperatorTok{/}\KeywordTok{sum}\NormalTok{(confusion_matrix.lasso[}\DecValTok{2}\NormalTok{,])}\OperatorTok{*}\DecValTok{100}
\NormalTok{pred.error_rate.lasso =}
\KeywordTok{sum}\NormalTok{(}\KeywordTok{diag}\NormalTok{(confusion_matrix.lasso[}\DecValTok{1}\OperatorTok{:}\DecValTok{2}\NormalTok{,}\DecValTok{2}\OperatorTok{:}\DecValTok{1}\NormalTok{]))}\OperatorTok{/}\KeywordTok{sum}\NormalTok{(confusion_matrix.lasso) }\OperatorTok{*}\DecValTok{100} \CommentTok{# proportion des mauvaises predictions}

\NormalTok{pred.accuracy.lasso}
\end{Highlighting}
\end{Shaded}

\begin{verbatim}
## [1] 76.62338
\end{verbatim}

\begin{Shaded}
\begin{Highlighting}[]
\NormalTok{pred.recall.lasso}
\end{Highlighting}
\end{Shaded}

\begin{verbatim}
## [1] 80.55556
\end{verbatim}

\begin{Shaded}
\begin{Highlighting}[]
\NormalTok{pred.specifity.lasso}
\end{Highlighting}
\end{Shaded}

\begin{verbatim}
## [1] 73.17073
\end{verbatim}

\begin{Shaded}
\begin{Highlighting}[]
\NormalTok{pred.error_rate.lasso}
\end{Highlighting}
\end{Shaded}

\begin{verbatim}
## [1] 23.37662
\end{verbatim}

Avec une performance globale de 76\% on predit les hauts prix à 80\% et
les as prix à 76\% Ce qui reste dans la gamme des performances des
regressions precedentes.

\hypertarget{courbe-roc-et-auc-1}{%
\subsection{Courbe ROC et AUC}\label{courbe-roc-et-auc-1}}

\begin{Shaded}
\begin{Highlighting}[]
\NormalTok{pred.lasso.min=}\KeywordTok{prediction}\NormalTok{(lasso.pred,tab.test}\OperatorTok{$}\NormalTok{medHousePriceBin)}
\NormalTok{perf.lasso.min=}\KeywordTok{performance}\NormalTok{(pred.lasso.min, }\StringTok{"tpr"}\NormalTok{ ,}\StringTok{"fpr"}\NormalTok{)}
\KeywordTok{plot}\NormalTok{(perf)}
\end{Highlighting}
\end{Shaded}

\includegraphics{TP3_files/figure-latex/unnamed-chunk-29-1.pdf} La
courbe est assez proche du coin supérieur gauche, ainsi le modèle semble
capturer le plus possible de vrais évènements avec le moins possible de
faux évènements.

\begin{Shaded}
\begin{Highlighting}[]
\NormalTok{ROC_auc.lasso=}\KeywordTok{performance}\NormalTok{(pred.lasso.min,}\StringTok{"auc"}\NormalTok{)}
\NormalTok{AUC.lasso=ROC_auc.lasso}\OperatorTok{@}\NormalTok{y.values[[}\DecValTok{1}\NormalTok{]]}
\KeywordTok{print}\NormalTok{(AUC.lasso)}
\end{Highlighting}
\end{Shaded}

\begin{verbatim}
## [1] 0.7686314
\end{verbatim}

l'AUC est de : 0.76

\hypertarget{e-k-folds-le-model-lasso}{%
\subsubsection{e) K-folds le model
lasso}\label{e-k-folds-le-model-lasso}}

\begin{Shaded}
\begin{Highlighting}[]
\CommentTok{##shuffling}
\KeywordTok{set.seed}\NormalTok{(}\DecValTok{1234}\NormalTok{)}
\NormalTok{rows <-}\StringTok{ }\KeywordTok{sample}\NormalTok{(}\KeywordTok{nrow}\NormalTok{(tabmed)) }\CommentTok{# melange des rows du dataset}
\NormalTok{tabmedk <-}\StringTok{ }\NormalTok{tabmed[rows, ]}
\CommentTok{## folds}
\NormalTok{k =}\StringTok{ }\DecValTok{15} \CommentTok{#as.integer(1/(1-r)) ## fold number}
\NormalTok{fold =}\StringTok{ }\KeywordTok{cut}\NormalTok{(}\KeywordTok{seq}\NormalTok{(}\DecValTok{1}\NormalTok{,}\KeywordTok{nrow}\NormalTok{(tabmed)), }\DataTypeTok{breaks =}\NormalTok{ k,}\DataTypeTok{labels =} \OtherTok{FALSE}\NormalTok{) }\CommentTok{#organisations des lignes du}
\CommentTok{#dataset en plusieurs fold}
\CommentTok{#initialisation des vectueurs qui va contenier les performances a chaque etape}
\NormalTok{accuracyk.lasso =}\StringTok{ }\KeywordTok{c}\NormalTok{()}
\NormalTok{recallk.lasso =}\StringTok{ }\KeywordTok{c}\NormalTok{()}
\NormalTok{precisionk.lasso =}\StringTok{ }\KeywordTok{c}\NormalTok{()}
\NormalTok{error_ratek.lasso =}\StringTok{ }\KeywordTok{c}\NormalTok{()}
\ControlFlowTok{for}\NormalTok{ (i }\ControlFlowTok{in} \DecValTok{1}\OperatorTok{:}\NormalTok{k) \{}
\NormalTok{  test_rows =}\StringTok{ }\KeywordTok{which}\NormalTok{(fold }\OperatorTok{==}\StringTok{ }\NormalTok{i,}\DataTypeTok{arr.ind =} \OtherTok{TRUE}\NormalTok{) }
\NormalTok{  tab.testk =}\StringTok{ }\NormalTok{tabmedk[test_rows,]}
\NormalTok{  tab.traink =}\StringTok{ }\NormalTok{tabmedk[}\OperatorTok{-}\NormalTok{test_rows,]}
\NormalTok{  X.traink =}\StringTok{ }\KeywordTok{as.matrix}\NormalTok{(tab.traink[,}\OperatorTok{-}\KeywordTok{dim}\NormalTok{(tab.traink)[}\DecValTok{2}\NormalTok{]])}
\NormalTok{  X.testk =}\StringTok{ }\KeywordTok{as.matrix}\NormalTok{(tab.testk[,}\OperatorTok{-}\KeywordTok{dim}\NormalTok{(tab.testk)[}\DecValTok{2}\NormalTok{]])}
\NormalTok{  Y.testk =}\StringTok{ }\NormalTok{tab.testk}\OperatorTok{$}\NormalTok{medHousePriceBin}
\NormalTok{  Y.traink =}\StringTok{ }\NormalTok{tab.traink}\OperatorTok{$}\NormalTok{medHousePriceBin}
  \CommentTok{### regression logistic }
\NormalTok{  model.lassok <-}\StringTok{ }\KeywordTok{glmnet}\NormalTok{(X.traink,Y.traink,}\DataTypeTok{alpha=}\DecValTok{1}\NormalTok{,}\DataTypeTok{lambda =}\NormalTok{ grid,}\DataTypeTok{family =} \StringTok{"binomial"}\NormalTok{)}
  \CommentTok{### prediction}
  \CommentTok{####################### cross validation}
\NormalTok{  lasso.cv.out<-}\KeywordTok{cv.glmnet}\NormalTok{(X.traink, Y.traink, }\DataTypeTok{alpha =} \DecValTok{1}\NormalTok{,}\DataTypeTok{nfolds =} \DecValTok{10}\NormalTok{,}\DataTypeTok{family =} \StringTok{"binomial"}\NormalTok{) }
  \CommentTok{# on sélectionne la meilleure valeur de lambda par validation croisée}
\NormalTok{  lasso.lamb.mink<-lasso.cv.out}\OperatorTok{$}\NormalTok{lambda.min }\CommentTok{# le meilleur lambda est celui qui produit the min MSE}
  
\NormalTok{  prob <-}\StringTok{ }\KeywordTok{predict}\NormalTok{(model.lassok, }\DataTypeTok{s =}\NormalTok{ lasso.lamb.mink, }\DataTypeTok{newx =}\NormalTok{ X.testk,}\DataTypeTok{type =} \StringTok{'response'}\NormalTok{)}
\NormalTok{  Y.pred.lasso =}\StringTok{  }\KeywordTok{as.integer}\NormalTok{(prob }\OperatorTok{>=}\StringTok{ }\NormalTok{Threshold) }
\NormalTok{  confusion_matrix =}\StringTok{ }\KeywordTok{table}\NormalTok{(Y.pred.lasso,tab.testk}\OperatorTok{$}\NormalTok{medHousePriceBin)}
\NormalTok{  accuracyk.lasso[i] =}\StringTok{ }\KeywordTok{sum}\NormalTok{(}\KeywordTok{diag}\NormalTok{(confusion_matrix))}\OperatorTok{/}\KeywordTok{sum}\NormalTok{(confusion_matrix)}\OperatorTok{*}\DecValTok{100}
\NormalTok{  recallk.lasso[i] =}\StringTok{ }\NormalTok{confusion_matrix[}\DecValTok{2}\NormalTok{,}\DecValTok{2}\NormalTok{]}\OperatorTok{/}\KeywordTok{sum}\NormalTok{(confusion_matrix[,}\DecValTok{2}\NormalTok{])}\OperatorTok{*}\DecValTok{100}
\NormalTok{  precisionk.lasso[i] =confusion_matrix[}\DecValTok{2}\NormalTok{,}\DecValTok{2}\NormalTok{]}\OperatorTok{/}\KeywordTok{sum}\NormalTok{(confusion_matrix[}\DecValTok{2}\NormalTok{,])}\OperatorTok{*}\DecValTok{100}
\NormalTok{  error_ratek.lasso[i] =}\StringTok{ }\KeywordTok{sum}\NormalTok{(}\KeywordTok{diag}\NormalTok{(confusion_matrix[}\DecValTok{1}\OperatorTok{:}\DecValTok{2}\NormalTok{,}\DecValTok{2}\OperatorTok{:}\DecValTok{1}\NormalTok{]))}\OperatorTok{/}\KeywordTok{sum}\NormalTok{(confusion_matrix) }\OperatorTok{*}\DecValTok{100} 
\NormalTok{\}}
\KeywordTok{mean}\NormalTok{(accuracyk.lasso)}
\end{Highlighting}
\end{Shaded}

\begin{verbatim}
## [1] 81.63139
\end{verbatim}

\begin{Shaded}
\begin{Highlighting}[]
\KeywordTok{mean}\NormalTok{(recallk.lasso)}
\end{Highlighting}
\end{Shaded}

\begin{verbatim}
## [1] 88.45316
\end{verbatim}

\begin{Shaded}
\begin{Highlighting}[]
\KeywordTok{mean}\NormalTok{(precisionk.lasso)}
\end{Highlighting}
\end{Shaded}

\begin{verbatim}
## [1] 78.3217
\end{verbatim}

\begin{Shaded}
\begin{Highlighting}[]
\KeywordTok{mean}\NormalTok{(error_ratek.lasso)}
\end{Highlighting}
\end{Shaded}

\begin{verbatim}
## [1] 18.36861
\end{verbatim}

\hypertarget{ii-conclusion}{%
\section{II )Conclusion}\label{ii-conclusion}}

On va comparer les differents modeles entre eux.

\begin{Shaded}
\begin{Highlighting}[]
\KeywordTok{par}\NormalTok{(}\DataTypeTok{mfrow =} \KeywordTok{c}\NormalTok{(}\DecValTok{2}\NormalTok{,}\DecValTok{2}\NormalTok{))}
\KeywordTok{boxplot}\NormalTok{(}\KeywordTok{data.frame}\NormalTok{(accuracyk.full,accuracyk.step,accuracyk.ridge,accuracyk.lasso),}\DataTypeTok{col =} \StringTok{"green"}\NormalTok{,}\DataTypeTok{boxwex =} \FloatTok{0.2}\NormalTok{, }\DataTypeTok{las =} \DecValTok{1}\NormalTok{,}\DataTypeTok{names =} \KeywordTok{c}\NormalTok{(}\StringTok{"full"}\NormalTok{,}\StringTok{"stepw"}\NormalTok{,}\StringTok{"ridge"}\NormalTok{,}\StringTok{"lasso"}\NormalTok{),}\DataTypeTok{main =} \StringTok{"accuracy"}\NormalTok{)}
\KeywordTok{boxplot}\NormalTok{(}\KeywordTok{data.frame}\NormalTok{(recallk.full,recallk.step,recallk.ridge,recallk.lasso),}\DataTypeTok{col =} \StringTok{"blue"}\NormalTok{,}\DataTypeTok{boxwex =} \FloatTok{0.2}\NormalTok{, }\DataTypeTok{las =} \DecValTok{1}\NormalTok{,}\DataTypeTok{names =} \KeywordTok{c}\NormalTok{(}\StringTok{"full"}\NormalTok{,}\StringTok{"stepw"}\NormalTok{,}\StringTok{"ridge"}\NormalTok{,}\StringTok{"lasso"}\NormalTok{),}\DataTypeTok{main =} \StringTok{"recall"}\NormalTok{)}
\KeywordTok{boxplot}\NormalTok{(}\KeywordTok{data.frame}\NormalTok{(precisionk.full,precisionk.step,precisionk.ridge,precisionk.lasso),}\DataTypeTok{col =} \StringTok{"yellow"}\NormalTok{,}\DataTypeTok{boxwex =} \FloatTok{0.2}\NormalTok{, }\DataTypeTok{las =} \DecValTok{1}\NormalTok{,}\DataTypeTok{names =} \KeywordTok{c}\NormalTok{(}\StringTok{"full"}\NormalTok{,}\StringTok{"stepw"}\NormalTok{,}\StringTok{"ridge"}\NormalTok{,}\StringTok{"lasso"}\NormalTok{),}\DataTypeTok{main =} \StringTok{"precision"}\NormalTok{)}
\KeywordTok{boxplot}\NormalTok{(}\KeywordTok{data.frame}\NormalTok{(error_ratek.full,error_ratek.step,error_ratek.ridge,error_ratek.lasso),}\DataTypeTok{col =} \StringTok{"red"}\NormalTok{,}\DataTypeTok{boxwex =} \FloatTok{0.2}\NormalTok{, }\DataTypeTok{las =} \DecValTok{1}\NormalTok{,}\DataTypeTok{names =} \KeywordTok{c}\NormalTok{(}\StringTok{"full"}\NormalTok{,}\StringTok{"stepw"}\NormalTok{,}\StringTok{"ridge"}\NormalTok{,}\StringTok{"lasso"}\NormalTok{),}\DataTypeTok{main =} \StringTok{"error rate"}\NormalTok{)}
\end{Highlighting}
\end{Shaded}

\includegraphics{TP3_files/figure-latex/unnamed-chunk-32-1.pdf} On
remarque, qu'en terme d'accuracy, le stepwise est meilleur. Et aussi
lorsqu'on regarde le recall la proportion des transactions à haut prix
bien prédites. La moyenne regression stepwise reste légèrement au dessus
des autres, avec une box bien plus courte, centrée autour de la moyenne.
Donc sur la base de ces 2 indicateurs (voir 3 car error\_rate =
100-accuracy), si le but est de prédire les transactions de prix
élévées(supérieur à la médiane), une régression stepwise est meilleur.

\end{document}
